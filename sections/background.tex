\chapter{Determination of the Background Contributions}
\label{chap:DeterminationoftheBackgroundContributions}
\begin{figure}[Hb]
	\centering
	\begin{subfigure}{0.4\textwidth}
		\includegraphics[width=\textwidth]{plots/bkg/wjets2.pdf}
		\caption{}
		\label{fig:bkg:fy_wjets1}
	\end{subfigure}
	\begin{subfigure}{0.4\textwidth}
		\includegraphics[width=\textwidth]{plots/bkg/wjets1.pdf}
		\caption{}
		\label{fig:bkg:fy_wjets2}
	\end{subfigure}
	\begin{subfigure}{0.4\textwidth}
		\includegraphics[width=0.75\textwidth]{plots/bkg/ttbar.pdf}
		\caption{}
		\label{fig:bkg:fy_ttbar}
	\end{subfigure}
	\caption[Feynman diagrams of the main background processes.]{\textbf{Feynman diagrams of the main background processes.} The dominant background contribution is given by W+jets production with quark (a) or quark-gluon final state (b) where the resulting jet is misidentified as originating from a W- or Z-boson. The second main background process is given by top pair production (c) with two real boosted W-bosons.}
\end{figure}
There are several processes with the same or similar final states as the diboson process which contribute to the overall background. The main contribution originates from W+jets production, where the W-boson decays leptonically and the quarks (or quark and gluon) in the final state are falsely reconstructed as W- or Z-boson-jets. Two example processes are shown in Fig.~\ref{fig:bkg:fy_wjets1} and Fig.~\ref{fig:bkg:fy_wjets2}. The second main contribution comes from top-pair (\ttbar) production with one leptonically and one hadronically decaying real W-boson as shown in Fig.~\ref{fig:bkg:fy_ttbar}. Additionaly, Standard Model diboson production, which includes processes with and without triple gauge coupling (Fig.~\ref{fig:bkg:fy_WWtgc}-\ref{fig:bkg:fy_WZSM}), are taken into account as minor backgrounds. The last minor contribution comes from single top production in association with a W-boson (Fig.~\ref{fig:bkg:fy_stoptw}) as well as in the s- (Fig.~\ref{fig:bkg:fy_stops}) and t-channel (Fig.~\ref{fig:bkg:fy_stopt}).\\
In this chapter, the analysis strategy for the determination of the background contributions is illustrated.  First, the procedure of extracting the background normalizations from the \Mpr -spectrum is described. Then the extraction of the background shapes in the \MWV -spectrum and the final background estimations for the sideband and signal region using the alpha-ratio-method are presented.

\begin{figure}
	\centering
	\begin{subfigure}{0.4\textwidth}
		\includegraphics[width=\textwidth]{plots/bkg/SMWW.pdf}
		\caption{}
		\label{fig:bkg:fy_WWtgc}
	\end{subfigure}
	\begin{subfigure}{0.4\textwidth}
		\includegraphics[width=\textwidth]{plots/bkg/SMWZ.pdf}
		\caption{}
		\label{fig:bkg:fy_WZtgc}
	\end{subfigure}
	\begin{subfigure}{0.4\textwidth}
		\includegraphics[width=\textwidth]{plots/bkg/SMWW2.pdf}
		\caption{}
		\label{fig:bkg:fy_WWSM}
	\end{subfigure}
	\begin{subfigure}{0.4\textwidth}
		\includegraphics[width=\textwidth]{plots/bkg/SMWZ2.pdf}
		\caption{}
		\label{fig:bkg:fy_WZSM}
	\end{subfigure}
	\begin{subfigure}{0.3\textwidth}
		\includegraphics[width=\textwidth]{plots/bkg/stoptW.pdf}
		\caption{}
		\label{fig:bkg:fy_stoptw}
	\end{subfigure}
	\begin{subfigure}{0.4\textwidth}
		\includegraphics[width=\textwidth]{plots/bkg/stops.pdf}
		\caption{}
		\label{fig:bkg:fy_stops}
	\end{subfigure}
	\begin{subfigure}{\textwidth}
		\centering
		\includegraphics[width=0.4\textwidth]{plots/bkg/stopt.pdf}
		\caption{}
		\label{fig:bkg:fy_stopt}
	\end{subfigure}

	
	\caption[Feynman diagrams of the minor background processes.]{\textbf{Feynman diagrams of the minor background processes.} Additional background contributions that are taken into account are WW-/WZ-production with (a/b) and without (c/d) triple gauge coupling as well as single top production in association with a W-boson (e), in the s-channel (f) and in the t-channel (g).}
\end{figure}

\clearpage
\section{Background Estimation Strategy}
The background estimation strategy orientates on a previous diboson resonance search described in \cite{resonancepas} with minor modifications. The basic idea is to use two observables, the pruned jet mass of the hadronically decaying boson \Mpr \ and the invariant mass of the diboson system \MWV, to perform a data driven background extraction. The normalizations of the main backgrounds, W+jets and \ttbar \ , are extracted from a fit in the \Mpr -distribution, while the minor backgrounds, single top and SM-diboson, are normalised to their cross section. The shape of the W+jets background is taken from data in the sideband region using the alpha-ratio-method described in section~\ref{sec:AlphaRatioMethod}. The shapes of \ttbar \ and single top production are taken from fits to MC simulation while the shape of the SM-diboson process is given by the signal model with all aTGC-parameters set to zero (see section \ref{sec:NormalizationandShapeoftheaTGCContribution}).

\section{Normalizations of the Background Distributions}
\label{sec:bkgnorms}
The normalizations of the main background contributions are extracted from the pruned jet mass. The shapes in this spectrum are extracted from fits of predefined functions to MC simulations. The main backgrounds W+jets and \ttbar \ production have a considerably different shape. The W+jets production shows a broad peak coming from QCD-jets, which is functionally described by
\begin{align}
F_{\rm W+jets}(x) &= e^{-a_0(x-a_1)/\left(a_2+a_0\right)} \cdot \frac{1 + {\rm Erf}\left((x-a_1)/(a_2-a_0) \right)}{2} ~ , \\
{\rm Erf(x)} &= \frac{2}{\sqrt{\pi}}\int_0^x e^{-t^2}{\rm dt} ~ .
\end{align}
The \ttbar \ distribution, however, has a peaking contribution coming from real boosted hadronically decaying W-bosons as well as a continuous contribution coming from events where some of the b-quark decay products are falsely reconstructed as part of the W-boson jet. The continuous part is modelled by a function similar to $F_{\rm W+jets}$, while the peaking distribution is given by a Gaussian function $G(x,\mu,\sigma)$
\begin{align}
F_{\ttbar}(x) &= b_0 \cdot e^{b_1 x} \cdot \frac{1 + {\rm Erf}((x-b_2)/b_3)}{2} + (1-b_0)\cdot G(x,\mu,\sigma) ~ , ~ b_0 \in [0,1] ~ .
\end{align}
The function describing the single top distribution is similar to $F_{\ttbar}$ with the continuous part being represented by a simple exponential function representing the tW-channel and the peaking contribution representing the t-channel
\begin{align}
F_{\rm singletop}(x) &= c_0\cdot e^{c_1 x} + (1-c_0)\cdot G(x,\mu,\sigma) ~ , ~ c_0 \in [0,1] ~ .
\end{align}
Finally, the functional form for diboson production consists of a sum of two double Gaussians, corresponding to WW and WZ events, which only differ in their mean values $\mu_{\rm W}$ and $\mu_{\rm Z}$. The numeric values defining the proportion of WW and WZ are estimated from the respective cross sections (\cite{WWxsec},\cite{WZxsec})
\begin{align}
F_{\rm diboson}(x) =& \ \ 0.74 \cdot F_{\rm WW}\left(x,\delta,\mu_{\rm W},\sigma,\Delta_\mu,\Delta_\sigma\right) + 0.26 \cdot F_{\rm WZ}\left(x,\delta,\mu_{\rm Z},\sigma,\Delta_\mu,\Delta_\sigma\right) \\
=& \ \ 0.74 \cdot \left(
\delta \cdot G(x,\mu_{\rm W},\sigma) + (1-\delta) \cdot G(x,\mu_{\rm W}+\Delta_\mu,\sigma\cdot\Delta_\sigma)
\right)  \nonumber \\
&+ 0.26 \cdot \left(
\delta \cdot G(x,\mu_{\rm Z},\sigma) + (1-\delta) \cdot G(x,\mu_{\rm Z}+\Delta_\mu,\sigma\cdot\Delta_\sigma)
\right) ~ , ~ \delta \in [0,1] ~ . 
\end{align}
These functions are now fitted to the corresponding MC simulation to obtain a template for each background contribution. The resulting fits to the MC simulations are shown in Fig.~\ref{fig:bkg:mjMC} for the electron and muon channel.

\begin{figure}
	\centering
	\begin{subfigure}{0.45\textwidth}
		\includegraphics[width=\textwidth]{plots/bkg/WJets_mj_el_ErfExp_with_pull.pdf}
	\end{subfigure}
	\begin{subfigure}{0.45\textwidth}
		\includegraphics[width=\textwidth]{plots/bkg/WJets_mj_mu_ErfExp_with_pull.pdf}
	\end{subfigure}
	\begin{subfigure}{0.45\textwidth}
		\includegraphics[width=\textwidth]{plots/bkg/TTbar_mj_el_2Gaus_ErfExp_with_pull.pdf}
	\end{subfigure}
	\begin{subfigure}{0.45\textwidth}
		\includegraphics[width=\textwidth]{plots/bkg/TTbar_mj_mu_2Gaus_ErfExp_with_pull.pdf}
	\end{subfigure}
	\begin{subfigure}{0.45\textwidth}
		\includegraphics[width=\textwidth]{plots/bkg/STop_mj_el_ExpGaus_with_pull.pdf}
	\end{subfigure}
	\begin{subfigure}{0.45\textwidth}
		\includegraphics[width=\textwidth]{plots/bkg/STop_mj_mu_ExpGaus_with_pull.pdf}
	\end{subfigure}
	\begin{subfigure}{0.45\textwidth}
		\includegraphics[width=\textwidth]{plots/bkg/VV_mj_el_2_2Gaus_with_pull.pdf}
	\end{subfigure}
	\begin{subfigure}{0.45\textwidth}
		\includegraphics[width=\textwidth]{plots/bkg/VV_mj_mu_2_2Gaus_with_pull.pdf}
	\end{subfigure}
	\caption[Background shapes in the pruned jet mass spectrum]{Background shapes in the pruned jet mass spectrum extracted from fit to MC simulation. The ratio plot shows the difference of the data and the fitted function normalised to the data uncertainty.}
	\label{fig:bkg:mjMC}
\end{figure}



\section{Shapes in the Sensitive Variable}
The final limit extraction is done in the invariant mass spectrum of the diboson system, \MWV \ . This variable is equal to the hard scale of the signal process. By applying a cut to this variable ($<3.5$\,TeV), a reference point for the scale of the new physics $\Lambda^2$ ($\gg3.5^2$\,TeV$^2$) is obtained. \\
The following section describes the background extraction for \MWV .

\subsection{Alpha-Ratio-Method}
\label{sec:AlphaRatioMethod}
The shape of the W+jets contribution in the signal region is estimated using a technique developed for diboson resonance searches, the so-called alpha-ratio-method (see e.g. \cite{resonancepas}). This method provides a way to determine a background shape in a signal region (SR) from data in a sideband region (SB) using a transfer function, the so-called alpha function. The main assumption is that there is a correlation between the SR- and SB-shapes that is modelled by the MC simulation. The alpha function is defined as
\begin{align}
\alpha_{\rm MC}(M_{WV})=F_{\rm MC}^{\rm SR}(M_{WV})/F_{\rm MC}^{\rm SB}(M_{WV}) \, ,
\end{align}
where $F_{\rm SR}^{\rm MC}$ and $F_{\rm SB}^{\rm MC}$ are the shapes extracted from MC simulation in the sideband and signal region, respectively. The background prediction in the signal region is then given by
\begin{equation}
F_{\rm prediction}^{\rm SR}(M_{WV})=\alpha_{\rm MC}(M_{WV})\cdot F_{\rm Data}^{\rm SB}(M_{WV}) \, ,
\end{equation}
where $F_{\rm Data}^{\rm SB}$ is the shape extracted from the sideband data.

\subsection{Shapes of the Minor Background Contributions}
To be able to apply the alpha-ratio-method for the background estimation of W+jets production, a fit to the data in the sideband region ($\Mpr \in [40,65]\cap[105,150]\,{\rm GeV}$) has to be done. Therefore, the shapes of the \ttbar , single top and diboson production have to be estimated, which is done by fitting predefined functions to the MC simulation. The used functions are given by
\begin{equation}
F_{\ttbar}^{\rm SB}(x) = e^{b_1x+\frac{b_2}{x}} \quad , \quad
F_{\rm singletop}^{\rm SB}(x) = e^{c_1x} \quad , \quad 
F_{\rm diboson}^{\rm SB}(x) = e^{d_1x} \, .
\end{equation}
The fit results are shown in Fig~\ref{fig:bkg:mwv_sb_minor}.
\dots
The shapes in the signal region are determined similarly using following functions
\begin{equation}
F_{\ttbar}^{\rm SB}(x) = e^{b_1x+\frac{b_2}{x}} \quad , \quad
F_{\rm singletop}^{\rm SB}(x) = e^{c_1x} \quad , \quad 
F_{\rm diboson}^{\rm SB}(x) = e^{d_1x+\frac{d_2}{x}} \, .
\end{equation}


\subsection{Shape of the W+jets Production}

%\subsection{Decorellation of the Main Background Functions}

\subsection{Closure Test of the Alpha-Ratio-Method}
