\begin{fmffile}{all_feynmans}
\setlength{\unitlength}{1mm}
\chapter{Determination of the Background Contributions}
\label{chap:DeterminationoftheBackgroundContributions}
There are several processes with the same or similar final states as the diboson process which contribute to the overall background. The main contribution originates from W+jets production, where the W-boson decays leptonically and the quarks (or quark and gluon) in the final state are falsely reconstructed as W- or Z-boson-jets. Two example processes are shown in Fig.~\ref{fig:bkg:fy_wjets1} and Fig.~\ref{fig:bkg:fy_wjets2}. The second main contribution comes from top-pair production with one leptonically and one hadronically decaying W-boson as shown in Fig.~\ref{fig:bkg:fy_ttbar}. Additionaly, single top production in association with a W-boson (Fig.~\ref{fig:bkg:fy_stoptw}) as well as in the s- (Fig.~\ref{fig:bkg:fy_stops}) and t-channel (Fig.~\ref{fig:bkg:fy_stopt}) are taken into account as minor backgrounds. The last minor contribution comes from Standard Model diboson production which includes processes without triple gauge coupling (Fig.~\ref{fig:bkg:fy_SMdiboson}).

\begin{figure}[hb]
	\centering
	\begin{subfigure}{0.4\textwidth}
		\includegraphics[width=\textwidth]{plots/bkg/wjets2.pdf}
		\caption{}
		\label{fig:bkg:fy_wjets1}
	\end{subfigure}
	\begin{subfigure}{0.4\textwidth}
		\includegraphics[width=\textwidth]{plots/bkg/wjets1.pdf}
		\caption{}
		\label{fig:bkg:fy_wjets2}
	\end{subfigure}
	\begin{subfigure}{0.4\textwidth}
		\includegraphics[width=0.75\textwidth]{plots/bkg/ttbar.pdf}
		\caption{}
		\label{fig:bkg:fy_ttbar}
	\end{subfigure}
	\caption[Feynman diagrams of the main background processes.]{Feynman diagrams of the main background processes. Shown are W+jets production with quark (a) and quark-gluon final state (b) as well as top pair production (c).}
\end{figure}

\begin{figure}
	\centering
	\begin{subfigure}{0.4\textwidth}
		\includegraphics[width=\textwidth]{plots/bkg/SMWW.pdf}
		\caption{}
	\end{subfigure}
	\begin{subfigure}{0.4\textwidth}
		\includegraphics[width=\textwidth]{plots/bkg/SMWZ.pdf}
		\caption{}
	\end{subfigure}
	\begin{subfigure}{0.4\textwidth}
		\includegraphics[width=\textwidth]{plots/bkg/SMWW2.pdf}
		\caption{}
	\end{subfigure}
	\begin{subfigure}{0.4\textwidth}
		\includegraphics[width=\textwidth]{plots/bkg/SMWZ2.pdf}
		\caption{}
	\end{subfigure}
	\begin{subfigure}{0.4\textwidth}
		\includegraphics[width=\textwidth]{plots/bkg/stoptW.pdf}
		\caption{}
		\label{fig:bkg:fy_stoptw}
	\end{subfigure}
	\begin{subfigure}{0.4\textwidth}
		\includegraphics[width=\textwidth]{plots/bkg/stops.pdf}
		\caption{}
		\label{fig:bkg:fy_stops}
	\end{subfigure}
	\begin{subfigure}{\textwidth}
		\centering
		\includegraphics[width=0.35\textwidth]{plots/bkg/stopt.pdf}
		\caption{}
		\label{fig:bkg:fy_stopt}
	\end{subfigure}

	
	\caption[Feynman diagrams of the minor background processes.]{Feynman diagrams of the minor background processes.single top production in association with a W-boson (a), in the s-channel (b) and in the t-channel (c).}
\end{figure}

\section{Background Estimation Strategy}


\section{Normalizations of the Background Distributions}


\section{Shapes in the Sensitive Variable}

\subsection{Alpha-Ratio-Method}

\subsection{Sideband Region}

\subsection{Signal Region}

\end{fmffile}