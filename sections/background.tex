\chapter{Determination of the Background Contributions}
\label{chap:DeterminationoftheBackgroundContributions}
\begin{figure}[H!b]
	\centering
	\begin{subfigure}{0.4\textwidth}
		\includegraphics[width=\textwidth]{plots/bkg/wjets2.pdf}
		\caption{}
		\label{fig:bkg:fy_wjets1}
	\end{subfigure}
	\begin{subfigure}{0.4\textwidth}
		\includegraphics[width=\textwidth]{plots/bkg/wjets1.pdf}
		\caption{}
		\label{fig:bkg:fy_wjets2}
	\end{subfigure}
	\begin{subfigure}{0.4\textwidth}
		\includegraphics[width=0.75\textwidth]{plots/bkg/ttbar.pdf}
		\caption{}
		\label{fig:bkg:fy_ttbar}
	\end{subfigure}
	\caption[Feynman diagrams of the main background processes.]{\textbf{Feynman diagrams of the main background processes.} The dominant background contribution is given by W+jets production with quark (a) or quark-gluon final state (b) where the resulting jet is misidentified as originating from a W- or Z-boson. The second main background process is given by top pair production (c) with two real boosted W-bosons.}
\end{figure}
There are several processes with the same or similar final states as the diboson process which contribute to the overall background. The main contribution originates from W+jets production, where the W-boson decays leptonically and the quarks (or quark and gluon) in the final state are falsely reconstructed as W- or Z-boson-jets. Two example processes are shown in Fig.~\ref{fig:bkg:fy_wjets1} and Fig.~\ref{fig:bkg:fy_wjets2}. The second main contribution comes from top-pair (\ttbar) production with one leptonically and one hadronically decaying real W-boson as shown in Fig.~\ref{fig:bkg:fy_ttbar}. Additionaly, Standard Model diboson production which includes processes with and without triple gauge coupling (Fig.~\ref{fig:bkg:fy_WWtgc}-\ref{fig:bkg:fy_WZSM}) are taken into account as minor backgrounds. The last minor contribution comes from single top production in association with a W-boson (Fig.~\ref{fig:bkg:fy_stoptw}) as well as in the s- (Fig.~\ref{fig:bkg:fy_stops}) and t-channel (Fig.~\ref{fig:bkg:fy_stopt}).\\
In this chapter, the analysis strategy for the determination of the background contributions is illustrated.  First, the procedure of extracting the background normalizations from the \Mpr -specturm is described. Then the extraction of the background shapes in the \MWV -spectrum and the final background estimations for the sideband and signal region are presented.

\begin{figure}
	\centering
	\begin{subfigure}{0.4\textwidth}
		\includegraphics[width=\textwidth]{plots/bkg/SMWW.pdf}
		\caption{}
		\label{fig:bkg:fy_WWtgc}
	\end{subfigure}
	\begin{subfigure}{0.4\textwidth}
		\includegraphics[width=\textwidth]{plots/bkg/SMWZ.pdf}
		\caption{}
		\label{fig:bkg:fy_WZtgc}
	\end{subfigure}
	\begin{subfigure}{0.4\textwidth}
		\includegraphics[width=\textwidth]{plots/bkg/SMWW2.pdf}
		\caption{}
		\label{fig:bkg:fy_WWSM}
	\end{subfigure}
	\begin{subfigure}{0.4\textwidth}
		\includegraphics[width=\textwidth]{plots/bkg/SMWZ2.pdf}
		\caption{}
		\label{fig:bkg:fy_WZSM}
	\end{subfigure}
	\begin{subfigure}{0.3\textwidth}
		\includegraphics[width=\textwidth]{plots/bkg/stoptW.pdf}
		\caption{}
		\label{fig:bkg:fy_stoptw}
	\end{subfigure}
	\begin{subfigure}{0.4\textwidth}
		\includegraphics[width=\textwidth]{plots/bkg/stops.pdf}
		\caption{}
		\label{fig:bkg:fy_stops}
	\end{subfigure}
	\begin{subfigure}{\textwidth}
		\centering
		\includegraphics[width=0.4\textwidth]{plots/bkg/stopt.pdf}
		\caption{}
		\label{fig:bkg:fy_stopt}
	\end{subfigure}

	
	\caption[Feynman diagrams of the minor background processes.]{\textbf{Feynman diagrams of the minor background processes.} Additional background contributions that are taken into account are WW-/WZ-production with (a/b) and without (c/d) triple gauge couling as well as single top production in association with a W-boson (e), in the s-channel (f) and in the t-channel (g).}
\end{figure}

\clearpage
\section{Background Estimation Strategy}
The background estimation strategy orientates on a previous diboson resonance described in \cite{resonancepas} with minor modifications. The normalizations of the main backgrounds, W+jets and \ttbar \ , are extracted from the The shape of the main background (W+jet production) is taken from data in the sideband region using the alpha-ratio-method described in section~\ref{sec:Al[haRatioMethod}. The shapes of \ttbar \ and single top production are taken from simulation. The shape of the Standard Model diboson process is taken from the signal model with all aTGC-parameters set to zero.

\section{Normalizations of the Background Distributions}
The normalizations of the background contributions are extracted from the \Mpr -spectrum. In this spectrum, the main backgrounds W+jets and top pair production have a considerably different shape. While the W+jets production [...]\\
The shapes in the \Mpr -spectrum are extracted from fits of predefined functions to Monte Carlo simulations. 

\section{Shapes in the Sensitive Variable}

\subsection{Alpha-Ratio-Method}
\label{sec:Al[haRatioMethod}
The alpha-ratio-method provides a way to determine a background shape in a signal region from data in a sideband region \cite{missing}.

\subsection{Sideband Region}

\subsection{Signal Region}
