\chapter{Limits on Anomalous Triple Gauge Couplings}
\label{chap::LimitsonATGCs}


\section{Limit Setting Procedure}
\subsection{Maximum Likelihood Estimation}
\subsection{Confidence Intervals}
\section{Systematic Uncertainties}
\label{sec:systematics}
\section{One-Dimensional Limits}
One-dimensional limits are computed by performing a scan over one of the aTGC parameters while keeping the remaining two set to zero. For each scanned point the likelihood is computed. Then, the delta-log-likelihood is formed, which is given by the difference of the likelihood of each point and the best fit point. The best fit point is the parameter value that maximizes the likelihood.  
\section{Two-Dimensional Limits}
\section{Limits in Vertex Parametrization}