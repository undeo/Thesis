\chapter{Experimental Setup}
\label{chap::ExperimentalSetup}
To be able to perform searches for new physics, particle collision at highest scales have to be analysed. The currently most advanced particle accelerator is the Large Hadron Collider, which is designed for a center of mass energy of $\sqrt{s}= 14$\,TeV. The particle collisions are recorded with two multy-purpose detectors, the Compact Muon Solenoid and the A Toroidal LHC Apparatus (ATLAS), a detector optimized for b-physics (LHCb) and a detector for heavy ion collisions (ALICE). The efforts of thousands of people working either at maintaining the accelerator and detectors or analysing the data have led to an improved understanding of the SM, most prominently with the discovery of the Higgs boson in 2012 \cite{cms_higgsdiscov,atlas_higgsdiscov}. The next goal of the LHC is to find proof of new physics that is not yet included in the SM, for which evidence has already been found (section~\ref{sec:aTGC}).\\

\noindent The following chapter outlines the Large Hadron Collider and the Compact Muon Solenoid, which provided the data used in this analysis. 
\section{The Large Hadron Collider (LHC)}
The Large Hadron Collider is a proton-proton synchrotron collider located in Geneva, Switzerland, at the European Organization for Nuclear Research (CERN). It is build inside the tunnel of its predecessor, the Large Electron Positron Collider (LEP).
\begin{figure}[b]
	\centering
	\includegraphics[width=\textwidth]{plots/expsetup/lhc.pdf}
	\caption[The large hadron collider]{The large hadron collider and the pre accelerators (taken from \cite{lhc_fig}).}
	\label{fig:expsetup:lhc}
\end{figure}
With a circumference of 26.7\,km \cite{LHC_DESIGN} and a peak center of mass energy of 13\,TeV it is the largest and most powerful hadronic particle accelerator up to date. A luminosity of the order of $10^{34}\,\rm{cm}^2\rm{s}$ \\

\noindent A total of 1232 dipol magnets are used to keep the particle beam on its track \cite{lhc_machine}. These magnets are given by superconducting coils cooled down to 2\,K, providing a magnetic field of 8.33\,T. About 3800 single aperture and 1000 twin aperture corrector magnets focus the beam and correct for higher order perturbations.

There are eight collision points (Point 1-8) along the beam with a detector positioned at four of them. The LHCb detector is a forward detector designed for high precision measurements in the B-sector \cite{LHCB}. The ALICE detector is 

\section{The Compact Muon Solenoid (CMS)}
The CMS detector is one of two multi-purpose detectors at the LHC. 
\begin{figure}
	\centering
	\includegraphics[width=\textwidth]{plots/expsetup/CMS_Slice.pdf}
	\caption[Transverse slice through the CMS detector]{Transverse slice through the CMS detector with exemplary particle tracks (taken from \cite{cms_slice}).}
	\label{fig:expsetup:cms_slice}
\end{figure}
\begin{description}
\item[Silicon Tracker] \hfill \\
\item[electromagnetic Calorimeter] \hfill \\
\item[Hadronic Calorimeter] \hfill \\
\item[Muon Tracker] \hfill \\
\item[Trigger System] \hfill \\
\item[Computing] \hfill \\
\end{description}