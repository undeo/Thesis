\chapter{Theoretical Introduction}
\label{chap::TheoreticalIntroduction}
The following chapter gives an introduction into the theoretical background relevant for this analysis. All derivations are taken from \cite{peskin,povh,mandl} if not stated otherwise. In the first section, the main properties of the Standard Model of particle physics is described. The second section focusses on physics beyond the Standard Model as well an approach to describe these new physics.

\section{The Standard Model of Particle Physics}
The Standard Model of particle physics (SM) describes the fundamental forces of nature (except for gravity) and their interactions with elementary particles: the strong force, the electromagnetic force and the weak force. Each force is represented by a quantum field theory, while the particles are excitation of quantum fields. The underlying symmetries of the interactions between particles are described by the symmetry groups SU(3)$\otimes$SU(2)$\otimes$U(1). According to Noether's theorem \cite{noether}, every symmetry leads to a conserved charge. In the SM, these charges are given by the electrical charge, the colour charge and the weak isospin. Each symmetry leads to one or more gauge bosons, which couple to these charges and are the mediators of the forces. \\
To understand the concept of quantum field theories, the Lagrangian formalism is needed, which was originally introduced to describe Newtonian mechanics. For every physical system a Lagrangian can be defined, which is given by
\begin{equation}
L(q,\dot{q},t) = T(\dot{q},t) - V(q,\dot{q},t) ~,
\end{equation}
where $T$ denotes the kinetic energy, $V$ the potential energy, $t$ the time, $q$ a generalized coordinate and $\dot{q}$ its derivative. The action $S$ is then defined as the time integral over the Lagrangian
\begin{equation}
S = \int \! L(q,\dot{q},t) \, \rm{dt} ~.
\end{equation}
Using the principle of least action, which states that in a stationary case the variation of the action $\delta S$ vanishes, one can derive the equations of motion from the Euler-Lagrange-equation
\begin{equation}
\frac{\rm{d}}{\rm{dt}}\frac{\partial L}{\partial \dot{q}} - \frac{\partial L}{\partial q} = 0 ~.
\end{equation}
To be able to apply the Lagrangian formalism in particle physics, the Lagrangian is replaced by the Lagrangian density $\mathcal{L}$, which is related to $L$ via
\begin{equation}
L=\int \! \mathcal{L} \, \mathrm{d}^3x ~.
\end{equation}
The generalized coordinates are replaced by fields and the time derivatives by the four-dimensional derivatives
\begin{align*}
q &\rightarrow \Phi ~, \\
\frac{\rm{d}}{\rm{dt}} &\rightarrow \frac{\partial}{\partial x^\mu} ~,
\end{align*}
with the four dimensional spacetime coordinates $x^\mu = (x^0,x^1,x^2,x^3)$, $x^0=t$. This leads to the Euler-Lagrange-equation for fields
\begin{equation}
\frac{\partial}{\partial x^\mu} \left( \frac{\partial \mathcal{L}}{\frac{\partial \Phi}{\partial x^\mu}} \right) - \frac{\partial \mathcal{L}}{\partial \Phi} = 0 ~. \label{eq:theo:eullag}
\end{equation}
Given the Lagrangian of a particle, this equation can be used to derived the corresponding equation of motion. For example, the Dirac-Lagrangian is given by
\begin{equation}
\mathcal{L}_{\rm Dirac}=i\hbar c \overline{\psi} \gamma^\mu \partial_\mu \psi -  mc^2\overline{\psi}\psi ~, \label{eq:theo:diraclag}
\end{equation}
with the gamma matrices $\gamma^\mu$, the Dirac-spinor $\psi$, its Dirac adjoint $\overline{\psi}=\psi^\dagger \gamma^0$ and the mass of the particle $m$. Inserting this Lagrangian into equation~\ref{eq:theo:eullag} leads to the equation of motion for a particle with mass $m$ and spin $J=\frac{1}{2}$, the so-called Dirac equation
\begin{equation}
\left(i\gamma^\mu\partial_\mu-m\right)\psi(x)=0 ~.
\end{equation}
This equation does not include any interactions. These are introduced by taking the gauge bosons (see section~\ref{subsec:theo:bosons}) into account.\\
With the recent detection of the Higgs boson in 2012 \cite{cms_higgsdiscov}\cite{atlas_higgsdiscov}, the last missing particle of the Standard Model has been observed.
The last remaining force, gravity, is not yet included in the SM, although there are attempts to add a gauge boson of gravitation, the graviton~\cite{}.\\



\subsection{Fermions}
A fermion is a particle with half-integer spin that therefore follows the Pauli exclusion principle \cite{pauli}, which states that two fermions with the same quantum number cannot exist in the same quantum state. The SM contains twelve elementary fermions with spin $J=\frac{1}{2}$ which are organized in two groups of six colour-neutral particles, the leptons, and six colour-charged particles, the quarks. The fermions can be arranged in three generations, each containing one electrically charged massive lepton, one neutral massless lepton (neutrino), one up-type quark with an electrical charge of $\frac{1}{3}e$ and one down-type quark with an electrical charge of $-\frac{2}{3}e$. The leptons or the quarks of one generation can be described with a SU(2) symmetry group as isospin-doublet with isospin $I=\frac{1}{2}$, with the third component of the isospin being $I_3=+\frac{1}{2}$ for the up-type and $I_3=-\frac{1}{2}$ for the down-type particle. The only difference between the different isospin-doublets is their mass, which is increasing with increasing generation number. The flavour eigenstates of the quarks $u,d,c,s,t$ and $b$ are not the eigenstates of the weak interaction, which are a mixture of these described by the CKM-matrix \cite{CKM}. This allows for quark decays between different generations. 
A summary of the fermions in the SM is shown in Tab.~\ref{tab:theo:fermions}.\\
\begin{table}[]
	\centering
	\caption[Summary of the fermions in the Standard Model]{Summary of the fermions in the Standard Model. Each generation of fermions contains a lepton- and a quark-isospin-doublet. Only the quarks carry a colour charge of red (r), green (g) or blue (b). Neutrinos only carry an isospin and thus only interact via the weak force.}
	\label{tab:theo:fermions}
	\begin{tabular}{crclrclrclccc}
	\hline
	& \multicolumn{9}{c}{Generation} & Electric & Colour &  weak \\
	& \multicolumn{3}{c}{I} & \multicolumn{3}{c}{II} & \multicolumn{3}{c}{II} & Charge ($e$) & Charge &  Isospin ($\hbar$) \\
	\hline
	\multirow{2}{*}{quarks} & \multirow{2}{*}{$\Big($} & $u$ & \multirow{2}{*}{$\Big)$} & \multirow{2}{*}{$\Big($} & $c$ & \multirow{2}{*}{$\Big)$} & \multirow{2}{*}{$\Big($} & $t$ & \multirow{2}{*}{$\Big)$} & +2/3 & r,g,b & +1/2 \\  
	 & & $d$ & & & $s$ & & & $b$ & & -1/3 & r,g,b & -1/2 \\ 
	\multirow{2}{*}{leptons}& \multirow{2}{*}{$\Big($} & $\nu_e$ & \multirow{2}{*}{$\Big)$} & \multirow{2}{*}{$\Big($} & $\nu_\mu$ & \multirow{2}{*}{$\Big)$} & \multirow{2}{*}{$\Big($} & $\nu_\tau$ & \multirow{2}{*}{$\Big)$} & - & - & +1/2 \\ 
	 & & $e$ & & & $\mu$ & & & $\tau$ & & -1 & - & -1/2 \\
	\hline
	\end{tabular}
\end{table}


\subsection{Gauge Bosons}
\label{subsec:theo:bosons}
\begin{table}[b]
	\centering
	\caption[Summary of the gauge bosons in the Standard Model]{Summary of the gauge bosons in the Standard Model. The values for the mass and charge of the photon are upper limits and are, as well as the other given masses, taken from \cite{SMmasses}.}
	\label{tab:theo:bosons}
	\begin{tabular}{cccc}
	\hline
	Boson & Interaction & Electric Charge $(e)$ & Mass (GeV/$c^2$) \\
	\hline
	$\gamma$ & electromagnetic & $<10^{-35}$ & $<10^{-27}$ \\
	gluon & strong & - & - \\
	Z & \multirow{2}{*}{weak} & - & $91.1876 \pm 0.0021$ \\
	W$^\pm$ & & $\pm 1$ & $80.385 \pm 0.015$ \\
	\hline
	\end{tabular}
\end{table}
The three fundamental forces in the SM are mediated by bosons with spin $J=1$, the so-called vector bosons. These bosons arise from gauge symmetries inherent to the quantum field theories describing the interactions. As an example, the derivation of the photon is portrayed in this section. A summary of all vector boson can be found in Tab.~\ref{tab:theo:bosons}.\\
The Lagrangian describing the fermions, i.e. the Dirac Lagrangian in equation~\ref{eq:theo:diraclag}, is required to be invariant under gauge transformations. Global gauge transformations, which transform the Dirac spinor according to
\begin{equation}
\psi\rightarrow e^{i\phi}\psi ~,
\end{equation}
trivially leave the equations of motion resulting from the Lagrangian unchanged and lead to charge conservation. Due to causality, the Lagrangian is also required to be invariant under local gauge transformations, which can be written as
\begin{equation}
\psi\rightarrow e^{i\phi(x)}\psi ~.
\end{equation} 
This transformation alters the Dirac Lagrangian
\begin{equation}
\mathcal{L}_{\rm Dirac} \rightarrow \mathcal{L}_{\rm Dirac} + \overline{\psi}\gamma^\mu\psi\partial_\mu \phi(x) 
\end{equation}
Invariance can be restored by replacing the ordinary derivative by the covariant derivative
\begin{equation}
\partial_\mu \rightarrow D_\mu = [\partial_\mu + iqA_\mu(x)] ~,
\end{equation}
with the charge $q$ and the gauge field $A_\mu$, which represents the massless photon.\\
The bosons transporting the strong force are given by the gluons, which are massless and couple to the colour charge. The theory describing gluons and their interaction with colour-charged particles is called quantum chromodynamics (QCD). Gluons can be introduced similarly to the photons by requiring the QCD-Lagrangian to be locally gauge invariant. Since gluons themselves carry a colour and an anti-colour, they can couple to each other as shown in Fig.~\ref{fig:theo:gluoncoupling}. There are three colour charges (red, green and blue) leading to eight different gluons.\\
\begin{figure}
	\centering
	\includegraphics[width=0.4\textwidth]{plots/theoIntro/feynmans/gluons.pdf}
	\includegraphics[width=0.4\textwidth]{plots/theoIntro/feynmans/gluons2.pdf}
	\caption[Feynman diagram of gluon self-interaction at tree-level]{Feynman diagram of gluon self-interaction at tree-level.}
	\label{fig:theo:gluoncoupling}
\end{figure}
For the weak force there are two different interactions. The neutral current interaction is mediated by the $Z$ boson and describes non-flavour-changing processes. The $Z$ boson has a mass of $M_Z=91.1976 \pm 0.0021$\,GeV \cite{SMmasses} and carries no electric charge. The charged current interactions are transported by the $W^\pm$ bosons, which carry an electric charge of $\pm e$. The $W$ boson has a mass of $80.385 \pm 0.015$\,GeV \cite{SMmasses}. These interactions can change the quark or lepton flavours. Since both the $Z$ and the $W$ boson are massive, they have a non-zero decay width which leads to a limited life time of $\tau_Z=2.6\cdot 10^{-25}$\,s and $\tau_W=3.2\cdot 10^{-25}$\,s, respectively. This causes a limited range and is the reason for the small strength of the weak force compared to the electromagnetic and strong force.\\ 
The SM also contains a scalar boson ($J=0$), the Higgs boson, which was recently discovered at the LHC \cite{cms_higgsdiscov,atlas_higgsdiscov}. The Higgs boson has a mass of $125.09 \pm 0.24$\,GeV and is responsible for the particle masses. The underlying theory is outlined in section~\ref{subsec:theo:EWK}.\\

\subsection{Electroweak Theory and Symmetry Breaking}
\label{subsec:theo:EWK}
Similar to the combination of the magnetic and electric forces to the electromagnetic force, the electromagnetic and the weak force can be combined at high scales to the electroweak force \cite{EWK} with two neutral gauge bosons, $\gamma , Z$, and two charged bosons $W^\pm$. The underlying symmetry is given by SU(2)$_I\otimes$U(1)$_Y$, where the SU(2)$_I$ group corresponds to the weak isospin and the $U(1)_Y$ group corresponds to the weak hypercharge, which is defined as
\begin{equation}
Y=2(Q-I_3) ~.
\end{equation} 
The $W^\pm$ bosons carry an isospin of $I=1$, with $I_3=+1$ for the $W^+$ and $I_3=-1$ for the $W^-$, respectively. Two new uncharged boson, the $W^0$ with $I=1,I_3=0$ and the $B^0$ with $I=0,I_3=0$, are introduced. The $\gamma$ and the $Z$ are orthogonal combinations of the $W^0$ and $B^0$ and are described as rotation by the Weinberg-angle $\theta_W$
\begin{equation}
\left( \begin{array}{c} \gamma \\ Z \end{array} \right) = \left( \begin{array}{c} \cos \theta_W ~  \sin \theta_W \\ -\sin \theta_W ~  \cos \theta_W \end{array} \right) \left( \begin{array}{c} B \\ W^3 \end{array} \right) ~,
\end{equation}
with $\theta_W=\arccos M_W/M_Z$.



, which is mediated by three bosons, $W^1,W^2$ and $W^3$. The $W^\pm$ bosons of the weak force are combinations of $W^1$ and $W^2$, given by
\begin{equation}
W^\pm=\frac{1}{\sqrt{2}}\left( W^1 \mp iW^2 \right) ~.
\end{equation}  
The U(1)$_Y$ group corresponds to the weak hypercharge, which is defined as

with the third component of the weak isospin $I_3$ and the electrical charge $Q$. The hypercharge is mediated by the $B$ boson. The $Z$ and $\gamma$ are orthogonal mixtures of the $B$ and the $W^3$




The simplest form of gauge boson self-interaction is the triple gauge coupling as portrayed in Fig.~\ref{fig:theo:tgc}. Neglecting C- and/or P-violating terms, the Lagrangian for this interaction is given by \cite{EFT}
\begin{align}
\mathcal{L} =& ig_{WWV}\Big( g_1^V(W_{\mu\nu}^+W^{-\mu} - W^{+\mu}W_{\mu\nu}^-)V^{\nu} + \kappa_VW_\mu^+W_\nu^-V^{\mu\nu}  + \frac{\lambda_V}{M_W^2}W_\mu^{\nu+}W_\nu^{-\rho}V_\rho^\mu  \Big) ~, \label{eq:theo:EWKlag}
\end{align}
where $V=Z,\gamma$; $X_{\mu\nu}=\partial_\mu X_\nu -\partial_\nu X_\mu$ and $X=W^\pm,\gamma ,Z$. The overall coupling constants $g_{WWV}$ are defined as $g_{WWZ} = -e \cot \theta_W$, with the Weinberg angle $\theta_W$, and  $g_{WW\gamma} =-e$. In the SM, the other coupling parameters are given by
\begin{align}
g_1^Z = g_1^\gamma = \kappa_Z = \kappa_\gamma &= 1 ~, \\
\lambda_Z = \lambda_\gamma &= 0 ~.
\end{align}
\begin{figure}
	\centering
	\includegraphics[width=0.45\textwidth]{plots/theoIntro/feynmans/TGC_WW.pdf}
	\caption[Feynman diagram of triple gauge coupling at tree level]{Feynman diagram of triple gauge coupling at tree level.}
	\label{fig:theo:tgc}
\end{figure}

\section{Anomalous Triple Gauge Couplings (aTGC)}
\label{sec:aTGC}
Although the SM has been confirmed to be self-consistent countless of times, there has been experimental prove that it is not a complete description of all fundamental interactions. All physics that is not included in the SM is call physics beyond the Standard Model (BSM). An example for BSM physics are neutrinos, which are massless particles according to the SM. However, by measuring the neutrino flux coming from the sun the SNO collaboration showed that neutrinos can oscillate and change their flavour \cite{SNO}. This is only possible if at least two of the neutrino mass eigenstates are unequal to zero \cite{}.\\
Another phenomenon that is not described by the SM is the so-called dark matter, which describes particles that only interact via gravitation and the weak force. An evidence for the existence of dark matter is given by the rotation speed of stars around the center of a galaxy. This speed is expected to increase with increasing distance from the bulk, where most of the galaxys mass is centered. The speed of stars beyond the bulk is expected to decrease with increasing distance. However, the observed rotation curves, as shown in Fig.~\ref{fig:theo:rotcurves}, show a constant rotation speed for stars beyond the galaxy bulk \cite{rotcurves}. This can be explained by additional mass with constant density in the galaxy, the so-called dark matter halo. One candidate for dark matter are neutrinos, which pass all requirements of a dark matter particle. However, current upper limits on their mass are too small to explain the observed rotation curves.
\begin{figure}
	\centering
	\includegraphics[width=0.6\textwidth]{plots/theoIntro/rotcurvesfig.png}
	\caption[Rotation curve of stars around the galaxy NGC 2198]{Rotation curve of stars around the galaxy NGC 2198. Shown are the measured rotation speeds of stars and gas clouds (dots) with a fitted function (straight line). Also shown are the components of the fitted function: visible matter (dashed line), gaseous matter (dotted line) and dark matter (dash-dotted line). In contrast to expectations, the rotation speeds stay constant at high distances.}
	\label{fig:theo:rotcurves}
\end{figure}

==gravity
==new particles -> W',Z',graviton\\
There are several attempts of new theories describing dark matter, e.g. supersymmetry \cite{SUSY}, that introduce new, heavy particles. At low energies (i.e. energies lower than the mass of the new particle), these particles can influence the SM couplings, like the self couplings of the electroweak bosons described in section~\ref{subsec:theo:EWK}. The new couplings are then called anomalous couplings, which can be described by altering the values of the couplings in the Lagrangian of the electroweak self interaction (eq.~\ref{eq:theo:EWKlag}), while $g_1^\gamma$ is fixed due to electromagnetic gauge invariance. This approach yields several problems. For example, the Lagrangian can be expanded by additional derivatives, which would only be suppressed by factors of $M_W^{-1}$. At energies above $M_W$, they would have to be taken into account, making the approach arbitrarily complex. The effective field theory approach, which is described in the next section, avoids these complications.



A model-independent approach of describing physics beyond the standard model that is currently not in experimental reach is the so-called effective field theory (EFT) \cite{EFT}. In this approach it is assumed that new physics exists at a scale $\Lambda$ which is much larger than the accessible energy. However, possible effects of new physics might affect processes at lower scales. These effects are described by integrating out the degrees of freedom of new heavy particles, leading to an effective Lagrangian with additional operators $\mathcal{O}^{(d)}$ of dimension $d\geq 5$
\begin{equation}
\mathcal{L}_{\rm eff} = \sum_{n=1}^{\infty} \sum_i \frac{c_i^{(n)}}{\Lambda^{n}}\mathcal{O}_i^{(n+4)} ~, \label{eq:theo:operators}
\end{equation}
where $c_i$ parametrizes the coupling strength of the new physics. Operators of dimension five are responsible for the majorana masses of neutrinos \cite{nu_majorana} and are of no interest for this analysis. There are many [HOW MANY?] dimensions six operators of which five introduce anomalous triple gauge couplings. Two of them are not C- and/or P-conserving and are not considered in this analysis. This leaves three operators, which can be chosen according to \cite{EFTparam} as
\begin{align}
\mathcal{O}_{WWW} &= \rm{Tr}[W_{\mu\nu}W^{\nu\rho}W^\mu_\rho] ~, \\
\mathcal{O}_{W} &=  (D_\mu\Phi)^\dagger W^{\mu\nu}(D_\nu\Phi) ~, \\
\mathcal{O}_{B} &= (D_\mu\Phi)^\dagger B^{\mu\nu}(D_\nu\Phi) ~.
\end{align}
Operators of dimension eight introduce anomalous quartic gauge couplings, but are suppressed by an additional factor of $\Lambda^{-2}$ and are therefore neglected. The dimension 6 operators affect the coupling parameters mentioned in section \ref{sec:aTGC} by introducing anomalous couplings
\begin{align}
g_1^Z &= 1 + c_W\frac{M_Z^2}{2\Lambda^2} ~, \\
\kappa_Z &= 1 + \left[ c_W - \sin^2\theta_W (c_W+c_B)\right] \frac{M_Z^2}{2\Lambda^2} ~, \\
\kappa_\gamma &= 1 + (c_W-c_B\tan^2\theta_W)\frac{M_W^2}{2\Lambda^2} ~, \\
\lambda_Z = \lambda_\gamma &= c_{WWW}\frac{3g^2M_W^2}{2\Lambda^2} ~, \label{eq:theo:vertpara}
\end{align}
where $g$ denotes the weak coupling constant. The strength of the aTGC's can now be parametrized by the three aTGC-parameters
\begin{align}
\frac{c_{WWW}}{\Lambda^2} ~ , \quad \frac{c_W}{\Lambda^2} ~ , \quad \frac{c_B}{\Lambda^2} ~.
\end{align}
The anomalous couplings introduced by these parameters add terms of the order $s^2/\Lambda^4$ to the diboson cross section \cite{EFT}, leading to an increased cross section at high scales. In Fig.~\ref{fig:theo:unitarity} the differential cross section for the SM WW process as well as the aTGC process corresponding to $\Tcwww = 1/(400 \, {\rm GeV})^{2}$ is shown in the invariant mass spectrum of the diboson system, which is equal to the scale of the process. Also shown is the unitarity bound, which is eventually violated by the aTGC process at high invariant masses. However, at this scale the approach is not valid anymore, since there is no justification to ignore the dimension eight operators in eq.~\ref{eq:theo:operators} at energies of the order of $\Lambda$. Arbitrarily many operators of even higher dimension would have to be taken into account, making the approach useless.\\
\begin{figure}
	\centering
	\includegraphics[width=\textwidth]{plots/theoIntro/unitarity.png}
	\caption[Comparison of the differential cross section of the SM and aTGC process]{Comparison of the differential cross section of the SM and aTGC process for $\cwww=\frac{1}{(400\,{\rm GeV})}$. The cross section increases dominantly at high invariant masses. Also shown is the unitarity bound. The picture is taken from \cite{EFT}.}
	\label{fig:theo:unitarity}
\end{figure}

%\section{Probability Density Functions (PDFs)}