\chapter{Theoretical Introduction}
\label{chap::TheoreticalIntroduction}
The following chapter gives an introduction into the theoretical background relevant for this analysis. First, the Standard Model of particle physics is described. Then, physics beyond the Standard Model as well as two approaches to describe these new physics are presented.

\section{The Standard Model of Particle Physics}
The Standard Model of particle physics (SM) describes the fundamental forces of nature (except for gravity) and their interactions with elementary particles: the strong force, the electromagnetic force and the weak force. Theoretical efforts by NAME proposed a combination of the electromagnetic and the weak force at high scales to the electroweak force \cite{EWK}. Each force is represented by a quantum field theory, where the underlying symmetries of the interactions are described by symmetry groups, namely SU(3) for the strong force and SU(2)$\otimes$U(1) for the electroweak force. Each gauge symmetry leads to one or more gauge bosons, which are the mediator of the forces. According to Noether's theorem \cite{noether}, every symmetry leads to a conserved \\
With the recent detection of the Higgs boson in 2012 \cite{cms_higgsdiscov}\cite{atlas_higgsdiscov}, the last missing particle of the Standard Model has been observed.
The last remaining force, gravity, is not yet included in the SM, although there are attempts to add a gauge boson of gravitation, the graviton~\cite{}.
\subsection{Fermions}
A fermion is a particle with half-integer spin that therefore follows the Pauli exclusion principle \cite{pauli}, which states that no two fermions with the same quantum number can exist in the same quantum state. The SM contains twelve elementary fermions with spin $J=\frac{1}{2}$ which are organized in two groups of six colour-neutral particles, the leptons, and six colour-charged particles, the quarks. The fermions can be arranged in three generation, each containing one electrically charged massive lepton, one neutral massless lepton (neutrino), one up-type quark with an electrical charge of $\frac{1}{3}e$ and one down-type quark with an electrical charge of $-\frac{2}{3}e$. The leptons or the quarks of one generation can be described as isospin-doublet with isospin $I=\frac{1}{2}$, with the third component of the isospin being $I_3=\frac{1}{2}$ for the up-type and $I_3=+\frac{1}{2}$ for the down-type particle. The only difference between the different isospin-doublets is their mass, which is increasing with increasing generation number. The flavour eigenstates of the quarks $u,d,c,s,t$ and $b$ are not the eigenstates of the weak interaction, which are a mixture of these described by the CKM-matrix \cite{CKM}. This allows for decays between different generations. 
A summary of the fermions in the SM is shown in Tab.~\ref{tab:theo:fermions}.\\
\begin{table}
	\centering
	\caption[Summary of the fermions in the Standard Model.]{Summary of the fermions in the Standard Model. Each generation of fermions contains a lepton- and a quark-isospin-doublet. Only the quarks carry a colour charge of red (r), green (g) or blue (b). Neutrinos only carry an isospin and thus only interact via the weak force.}
	\label{tab:theo:fermions}
	\begin{tabular}{crclrclrclccc}
	\hline
	& \multicolumn{9}{c}{Generation} & Electric & Colour &  weak \\
	& \multicolumn{3}{c}{I} & \multicolumn{3}{c}{II} & \multicolumn{3}{c}{II} & Charge ($e$) & Charge &  Isospin ($\hbar$) \\
	\hline
	\multirow{2}{*}{quarks} & \multirow{2}{*}{$\Big($} & $u$ & \multirow{2}{*}{$\Big)$} & \multirow{2}{*}{$\Big($} & $c$ & \multirow{2}{*}{$\Big)$} & \multirow{2}{*}{$\Big($} & $t$ & \multirow{2}{*}{$\Big)$} & +2/3 & r,g,b & +1/2 \\  
	 & & $d$ & & & $s$ & & & $b$ & & -1/3 & r,g,b & -1/2 \\ 
	\multirow{2}{*}{leptons}& \multirow{2}{*}{$\Big($} & $\nu_e$ & \multirow{2}{*}{$\Big)$} & \multirow{2}{*}{$\Big($} & $\nu_\mu$ & \multirow{2}{*}{$\Big)$} & \multirow{2}{*}{$\Big($} & $\nu_\tau$ & \multirow{2}{*}{$\Big)$} & - & - & +1/2 \\ 
	 & & $e$ & & & $\mu$ & & & $\tau$ & & -1 & - & -1/2 \\
	\hline
	\end{tabular}
\end{table}


\subsection{Gauge Bosons}
Any interaction in the SM is mediated by bosons with spin $J=1$, the so-called vector bosons. These bosons arise from gauge symmetries inherent to the quantum field theory describing the interactions. A summary of all vector boson is shown in Tab.~\ref{tab:theo:bosons}.\\
The electromagnetic force is mediated by photons ($\gamma$), which are massless particles. Therefore, the range of the electromagnetic force is infinite. The photon couples to the electric charge.\\
The bosons transporting the strong force are given by the gluons. Gluons are massless and couple to the colour charge. Since gluons themselves carry a colour charge and a colour anti charge, they can couple to each other. There are three colour charges (red, green and blue) leading to eight different gluons.\\
For the weak force there are two different interactions. The neutral current interaction is mediated by the $Z$ boson and describes non-flavour-changing processes. The $Z$ boson has a mass of $M_Z=91.1976 \pm 0.0021$\,GeV \cite{SMmasses} and carries no electric charge. The charged current interactions are transported by the $W^\pm$ bosons, which carry an electric charge of $\pm e$. The $W$ boson has a mass of $80.385 \pm 0.015$\,GeV. Since both the $Z$ and the $W$ boson are massive, they have a non-zero decay width which leads to a limited life time of $\tau_Z=2.6\cdot 10^{-25}$\,s and $\tau_W=3.2\cdot 10^{-25}$\,s, respectively. This causes a limited range of the weak force and is the reason for the small strength of the weak force compared to the electromagnetic and strong force.\\ 
The SM also contains a scalar boson ($J=0$), the Higgs boson, which was recently discovered at the LHC \cite{cms_higgsdiscov},\cite{atlas_higgsdiscov}. The Higgs boson has a mass of $125.09 \pm 0.24$\,GeV and is responsible for the particle masses.\\

\begin{table}
	\centering
	\caption[Summary of the gauge bosons in the Standard Model.]{Summary of the gauge bosons in the Standard Model. The values for the mass and charge of the photon are upper limits and are, as well as the other given masses, taken from \cite{SMmasses}.}
	\label{tab:theo:bosons}
	\begin{tabular}{cccc}
	\hline
	Boson & Interaction & Electric Charge $(e)$ & Mass (GeV/$c^2$) \\
	\hline
	$\gamma$ & electromagnetic & $<10^{-35}$ & $<10^{-27}$ \\
	gluon & strong & - & - \\
	Z & \multirow{2}{*}{weak} & - & $91.1876 \pm 0.0021$ \\
	W$^\pm$ & & $\pm 1$ & $80.385 \pm 0.015$ \\
	\hline
	\end{tabular}
\end{table}
\subsection{Lagrangian Formalism}
The Lagrangian formalism provides a way to describe Newtonian mechanics. However, it is also used in quantum field theory. For every physical system a Lagrangian can be defined, which is given by
\begin{equation}
L(q,\dot{q},t) = T(\dot{q},t) - V(q,\dot{q},t) ~,
\end{equation}
where $T$ denotes the kinetic energy, $V$ the potential energy, $t$ the time, $q$ a generalized coordinate and $\dot{q}$ its derivative. The action $S$ is now defined as the time integral over the Lagrangian
\begin{equation}
S = \int \! L(q,\dot{q},t) \, \rm{dt} ~.
\end{equation}
Using the principle of least action, which states that in a stationary case the variation of the action $\delta S$ vanishes, one can derive the equations of motion from the Euler-Lagrange-equation
\begin{equation}
\frac{\rm{d}}{\rm{dt}}\frac{\partial L}{\partial \dot{q}} - \frac{\partial L}{\partial q} = 0 ~.
\end{equation}
To be able to apply the Lagrangian formalism in particle physics, the Lagrangian is replaced by the Lagrangian density $\mathcal{L}$, which is related to $L$ via
\begin{equation}
L=\int \! \mathcal{L} \, \mathrm{d}^3x ~.
\end{equation}
The generalized coordinates are replaced by fields and the time derivatives by the four-dimensional derivatives
\begin{align*}
q &\rightarrow \Phi ~, \\
\frac{\rm{d}}{\rm{dt}} &\rightarrow \frac{\rm{d}}{\rm{d}x^\mu} ~,
\end{align*}
with the four dimensional spacetime coordinates $x^\mu = (x^0,x^1,x^2,x^3)$, $x^0=t$. This leads to the new Euler-Lagrange-equation
\begin{equation}
\frac{\partial}{\partial x^\mu} \left( \frac{\partial \mathcal{L}}{\frac{\partial \Phi}{\partial x^\mu}} \right) - \frac{\partial \mathcal{L}}{\partial \Phi} = 0 ~.
\end{equation}


\subsection{Quantum Field Theories}
\subsection{Electroweak Theory}

\subsection{Quantum Chromodynamics (QCD)}
The theory of quantum chromodynamics describes the interaction between colour-charged particles. The interaction is mediated by gluons, which are massless, colour-charged vector bosons. Each gluon carries a colour and an anti-colour
\section{Physics Beyond the Standard Model}
Although the SM has been confirmed to be self-consistent countless of times, there has been experimental prove that it is not a complete description of all fundamental interactions. For example, in the SM neutrinos are massless particles. However, by measuring the neutrino flux coming from the sun the SNO collaboration showed that neutrinos can oscillate and change their flavour \cite{SNO}. This is only possible if at least two of the neutrino mass eigenstates are unequal to zero \cite{}. While including neutrino masses into the SM is a rather small correction \cite{numassesSMP}, there is also 
==gravity
==dark matter
==new particles -> W',Z',graviton
\section{Anomalous Triple Gauge Couplings (aTGC)}
\label{sec:aTGC}
\begin{figure}[b]
	\centering
	\includegraphics[width=0.35\textwidth]{plots/theoIntro/feynmans/TGC_WW.pdf}
	\caption[Feynman diagram of triple gauge coupling at tree level.]{Feynman diagram of triple gauge coupling at tree level.}
	\label{fig:theo:tgc}
\end{figure}
Since the gauge boson of the weak force carry an isospin, they can interact with each other. The simplest form of gauge boson self-interaction is the triple gauge coupling as portrayed in Fig.~\ref{fig:theo:tgc}. Neglecting C- and/or P-violating terms, the Lagrangian for this interaction is given by \cite{EFT}
\begin{align}
\mathcal{L} =& ig_{WWV}\Big( g_1^V(W_{\mu\nu}^+W^{-\mu} - W^{+\mu}W_{\mu\nu}^-)V^{\nu} + \kappa_VW_\mu^+W_\nu^-V^{\mu\nu}  + \frac{\lambda_V}{M_W^2}W_\mu^{\nu+}W_\nu^{-\rho}V_\rho^\mu  \Big) ~,
\end{align}
where $V=Z,\gamma$; $X_{\mu\nu}=\partial_\mu X_\nu -\partial_\nu X_\mu$ and $X=W^\pm,\gamma ,Z$. The overall coupling constants $g_{WWV}$ are defined as $g_{WWZ} = -e \cot \theta_W$, with the Weinberg angle $\theta_W$, and  $g_{WW\gamma} =-e$. In the SM, the other coupling parameters are given by
\begin{align}
g_1^Z = g_1^\gamma = \kappa_Z = \kappa_\gamma &= 1 ~, \\
\lambda_Z = \lambda_\gamma &= 0 ~.
\end{align}
Anomalous couplings can now be described by altering the values of these couplings, while $g_1^\gamma$ is fixed due to electromagnetic gauge invariance. This approach yields several problems. For example, the Lagrangian can be expanded by additional derivatives, which would only be suppressed by factors of $M_W^{-1}$. At energies above $M_W$, they would have to be taken into account, making the approach arbitrarily complex. The effective field theory approach, which is described in the next section, avoid these complications.\\ 
\section{Effective Field Theory (EFT)}
A model-independent approach of describing physics beyond the standard model that is currently not in experimental reach is the so-called effective field theory (EFT) \cite{EFT}. In this approach it is assumed that new physics exists at a scale $\Lambda$ which is much larger than the accessible energy. However, possible effects of new physics might affect processes at lower scales. These effects are described by integrating out the degrees of freedom of new heavy particles, leading to an effective Lagrangian with additional operators $\mathcal{O}^{(d)}$ of dimension $d\geq 5$
\begin{equation}
\mathcal{L}_{\rm eff} = \sum_{n=1}^{\infty} \sum_i \frac{c_i^{(n)}}{\Lambda^{n}}\mathcal{O}_i^{(n+4)} ~,
\end{equation}
where $c_i$ parametrizes the coupling strength of the new physics. Operators of dimension five are responsible for the majorana masses of neutrinos \cite{nu_majorana} and are of no interest for this analysis. There are many [HOW MANY?] dimensions six operators of which five introduce anomalous triple gauge couplings. Two of them are not C- and/or P-conserving and are not considered in this analysis. This leaves three operators, which can be chosen according to \cite{EFTparam} as
\begin{align}
\mathcal{O}_{WWW} &= \rm{Tr}[W_{\mu\nu}W^{\nu\rho}W^\mu_\rho] ~, \\
\mathcal{O}_{W} &=  (D_\mu\Phi)^\dagger W^{\mu\nu}(D_\nu\Phi) ~, \\
\mathcal{O}_{B} &= (D_\mu\Phi)^\dagger B^{\mu\nu}(D_\nu\Phi) ~.
\end{align}
Operators of dimension eight introduce anomalous quartic gauge couplings, but are suppressed by an additional factor of $\Lambda^{-2}$ and are therefore neglected. The dimension 6 operators affect the coupling parameters mentioned in section \ref{sec:aTGC} by introducing anomalous couplings
\begin{align}
g_1^Z &= 1 + c_W\frac{M_Z^2}{2\Lambda^2} ~, \\
\kappa_Z &= 1 + \left[ c_W - \sin^2\theta_W (c_W+c_B)\right] \frac{M_Z^2}{2\Lambda^2} ~, \\
\kappa_\gamma &= 1 + (c_W-c_B\tan^2\theta_W)\frac{M_W^2}{2\Lambda^2} ~, \\
\lambda_Z = \lambda_\gamma &= c_{WWW}\frac{3g^2M_W^2}{2\Lambda^2} ~, \label{eq:theo:vertpara}
\end{align}
where $g$ denotes the weak coupling constant. The strength of the aTGC's can now be parametrized by the three aTGC-parameters
\begin{align}
\frac{c_{WWW}}{\Lambda^2} \quad , \quad \frac{c_W}{\Lambda^2} \quad , \quad \frac{c_B}{\Lambda^2} ~.
\end{align} 
In contrast to the Lagrangian approach presented in section~\ref{sec:aTGC}, the EFT approach is valid up to a scale $\Lambda$, which is (by definition) higher than the scale of the considered processes.\\


%\section{Probability Density Functions (PDFs)}