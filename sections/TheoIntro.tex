\chapter{Theoretical Introduction}
\label{chap::TheoreticalIntroduction}

\section{The Standard Model of Particle Physics}
\subsection{Fermions}
\subsection{Gauge Bosons}
\subsection{Electroweak Theory}
\subsection{Quantum Chromodynamics (QCD)}
\section{Physics Beyond the Standard Model}
Although the SMP has been confirmed countless of times, there has been experimental prove that it is not complete. For example, in the SMP neutrinos are massles particles. However, by measuring the neutrino flux coming from the sun the SNO collaboration showed that neutrinos can oscillate and change their flavor \cite{SNO}. This is only possible if at least two of the neutrino mass eigenstates are uneqal to zero \cite{}. While including neutrino masses into the SMP is a rather small correction \cite{numassesSMP}, there is also 
==gravity
==dark matter
==new particles -> W',Z'
\section{Anomalous Triple Gauge Couplings (aTGC)}
==TGC, channels
==aTGC, effects
\section{Effective Field Theory (EFT)}
==field theory
A model-independet approach of describing physics beyond the standard model that is currently not in experimental reach is the so-called effective field theroy (EFT) \cite{EFT}. In this approach it is assumed that new physics exists at a scale $\Lambda$ which is much larger than the accesible energy. However, possible effects of the new physics might be visible at lower scales by affecing event yields or shapes. These effects are parametrized by expanding the SM Langrangian by additional operators. The SM Lagrangian only contains operators up to dimension four \cite{}, while the new operators are of dimension $d>4$
\begin{equation}
L_{EFT} = L_{SM} + \sum_{n>4} \sum_i \frac{c_i^n}{\Lambda^2}O^{(n)}() ~.
\end{equation}
Operators of dimension five are responsible for the majorana masses of neutrinos \cite{nu_majo} and are of no interest for this analysis. There are many [HOW MANY?] dimensions six operators of which five introduce anomalous triple gauge couplings. Two of them are not C- and/or P-conserving and are not considered in this analysis. This leaves three operators, which are given by
\begin{align}
O_{c_{WWW}} &= \frac{c_{WWW}}{/Lambda^2}\\
O_{c_W} &= \frac{c_W}{/Lamda^2} \\
O_{c_B} &= /frac}{c_B}{/Lambda^2} ~.
\end{align}
\section{Probability Density Functions (PDFs)}