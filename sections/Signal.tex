\chapter{Modelling of the Signal Process}
\label{chap:ModellingoftheSignalProcess}

\section{Monte Carlo Samples}
\subsection{\dots}
\subsection{\dots}
\section{WW- and WZ-category}
The signal process under consideration contains both WW and WZ events, which are expected to have a peaking distribution in the \Mpr -spectrum at the W or the Z mass, respectively. Due to the limited resolution of the hadronic calorimeter those two peaks cannot be easily distinguished. However, if the signal region is divided into a low-mass- ($\Mpr \in [65,85]\,{\rm GeV}$) and a high-mass-region ($\Mpr \in [85,105]\,{\rm GeV}$), their compositions differ significantly. While the low-mass-region contains most of the WW-events and some WZ-events with low \Mpr \ , the high-mass-region contains most of the WZ-events and some WW-events with high \Mpr .\\
The division of the signal region provides some sensitivity to different aTGC scenarios. Two of the aTGC parameters, $c_{WWW}$ and $c_W$, enhance the WW and WZ production roughly equally. However, $c_B$ almost exclusively contributes to WW-events. 
\section{Normalization and Shape of the aTGC Contribution}
\label{sec:NormalizationandShapeoftheaTGCContribution}
\section{Uncertainties on the Slope Parameters}
\section{Interference with the Standard Model Process}
\section{Generator Level Studies}
\subsection{Interference Between Different Anomalous Coupling Contributions}
\subsection{Standard Model Interference at Low aTGC Values}
