\chapter{Modelling of the Signal Process}
\label{chap:signal}
In this chapter 
\section{Monte Carlo Samples}
\dots \\
For each aTGC parameter two working points are defined, which are summarized in table~\ref{tab:signal:aTGCpoints}. The MC samples for these working points are obtained by reweighting the originally created samples.
\begin{table}
	\centering
	\caption{\dots}
	\label{tab:signal:aTGCpoints}
	\begin{tabular}{cccc}
	\hline
	\# & \Tcwww & \Tccw & \Tcb \\
	\hline
	1,2 &  $\pm 12$ & 0 & 0\\
	3,4 & 0 &  $\pm 20$ & 0\\
	5,6 & 0 & 0 &  $\pm 60$\\
	7,8 &	$\pm 12$ & $\pm 20$ & $\pm 60$\\
	\hline
	\end{tabular}
\end{table} 
\subsection{\dots}
\subsection{\dots}
\section{Division of the Signal Region into WW- and WZ-category}
The signal process under consideration contains both WW and WZ events, which are expected to have a peaking distribution in the \Mpr -spectrum at the W or the Z mass, respectively. Due to the limited resolution of the hadronic calorimeter, those two peaks cannot be easily distinguished. However, if the signal region is divided into a low-mass- ($\Mpr \in [65,85]\,{\rm GeV}$) and a high-mass-region ($\Mpr \in [85,105]\,{\rm GeV}$), their compositions differ significantly, as can be seen in figure~\ref{fig:signal:wwwz_comp}. While the low-mass-region contains most of the WW-events and some WZ-events with low \Mpr \ , the high-mass-region contains most of the WZ-events and some WW-events with high \Mpr .\\
The division of the signal region provides some sensitivity to different aTGC scenarios. Two of the aTGC parameters, $c_{WWW}$ and $c_W$, enhance the WW and WZ production roughly equally. However, $c_B$ almost exclusively contributes to WW-events. This leads to differentiation power between two scenarios. In the first one, only $c_B$ is unequal to zero, which would result in a larger enhancement and thus in looser limits in the WW-category. In the second scenario, only $c_{WWW}$ and $c_W$ are unequal to zero, leading to roughly similar limits in both categories.
\begin{figure}
	\centering
	\begin{subfigure}{0.45\textwidth}
		\includegraphics[width=\textwidth]{plots/signal/wwwz_SM_mu.pdf}
		\caption{}
	\end{subfigure}
	\begin{subfigure}{0.45\textwidth}
		\includegraphics[width=\textwidth]{plots/signal/wwwz_cwww_mu.pdf}
		\caption{}
	\end{subfigure}
	\begin{subfigure}{0.45\textwidth}
		\includegraphics[width=\textwidth]{plots/signal/wwwz_ccw_mu.pdf}
		\caption{}
	\end{subfigure}
	\begin{subfigure}{0.45\textwidth}
		\includegraphics[width=\textwidth]{plots/signal/wwwz_cb_mu.pdf}
		\caption{}
	\end{subfigure}
	\caption[Comparison of the WW and WZ MC signal sample in the \Mpr -spectrum in the muon channel]{Comparison of the WW and WZ MC signal sample in the \Mpr -spectrum in the muon channel for different scenarios: SM (a), $\cwww=12$\,TeV$^{-2}$ (b), $\ccw=20$\,TeV$^{-2}$ (c) and $\cb=60$\,TeV$^{-2}$ (d). }
	\label{fig:signal:wwwz_comp}
\end{figure}

\section{Normalization and Shape of the aTGC Contribution}
The signal function is given by an analytical function depending on the three aTGC parameters. This allows for different scenarios, where either one, two or three aTGC parameters are unequal to zero. The signal model can also be easily reparametrised to the vertex parametrization used by previous analyses.\\
The final normalization of the SM diboson process is estimated by its cross section. The normalization of the aTGC contribution is then obtained by effectively applying a scaling factor that is extracted from the LO signal MC signal sample. Since the event yields scale quadratically in the aTGC parameters, the scaling factor is defined as
\begin{equation}
S_{\rm aTGC} = 1 + \left[ \Delta f_1(c_{WWW}) + \Delta f_2(c_{W}) + \Delta f_3(c_B) \right] ~,
\end{equation}
where the $\Delta f_i$ are given by
\begin{align}
\Delta f_i(x) &= f_i(x)-1 ~, \\
f_i(x) &= a_i + b_i x + d_i x^2 ~. \label{eq:signal:scale}
\end{align}
The functions $f_i(x)$ describe the relative increase in event yields for each aTGC parameter. This definition ensures $S_{\rm aTGC}=1$ for the SM case. The parameters of these functions are extracted from fits to the aTGC yields normalized to the SM yields for the different working points as shown in figure~\ref{fig:signal:atgcyields_mu} for the muon channel (see figure~\ref{fig:app:atgcyields_el} for the electron channel).
\begin{figure}
	\centering
	\begin{subfigure}{\textwidth}
		\includegraphics[width=0.5\textwidth]{plots/signal/yields_cwww_WW_mu.pdf}
		\includegraphics[width=0.5\textwidth]{plots/signal/yields_cwww_WZ_mu.pdf}
	\end{subfigure}
	\begin{subfigure}{\textwidth}
		\includegraphics[width=0.5\textwidth]{plots/signal/yields_ccw_WW_mu.pdf}
		\includegraphics[width=0.5\textwidth]{plots/signal/yields_ccw_WZ_mu.pdf}
	\end{subfigure}
	\begin{subfigure}{\textwidth}
		\includegraphics[width=0.5\textwidth]{plots/signal/yields_cb_WW_mu.pdf}
		\includegraphics[width=0.5\textwidth]{plots/signal/yields_cb_WZ_mu.pdf}
	\end{subfigure}
	\caption[Relative yields of the aTGC contributions in the muon channel.]{Relative yields of the aTGC contributions in the muon channel for the WW-category (left) and the WZ-category (right). The red line shows the fitted quadratic functions, which are used to normalize the signal contribution.}
	\label{fig:signal:atgcyields_mu}
\end{figure}
As can be seen here, the yields for positive and negative aTGC parameter values differ, which is caused by interference effects between the aTGC and the SM contribution. This is taken into account by the linear terms in equation~\ref{eq:signal:scale}.\\


\label{sec:NormalizationandShapeoftheaTGCContribution}
\section{Uncertainties on the Slope Parameters}
%\section{Interference with the Standard Model Process}
\section{Generator Level Studies}
The available MC sample with complete detector simulation only contains the aTGC working points mentioned in table~\ref{tab:signal:aTGCpoints}. To be able to model interference effects between different aTGC contributions, samples with two aTGC parameters set to non-zero values are needed. Therefore, the MC generation is redone for a total of 150 new working points, leaving out the complete detector simulation. This includes small aTGC parameter values, where the interference with the SM process is expected to be most visible. This enables for a verification of the modelling of the SM interference effects.
\subsection{Interference Between Different Anomalous Coupling Contributions}

\subsection{Standard Model Interference at Low aTGC Values}
