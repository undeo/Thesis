%% LaTeX2e class for student theses
%% sections/apendix.tex
%% 
%% Karlsruhe Institute of Technology
%% Institute for Program Structures and Data Organization
%% Chair for Software Design and Quality (SDQ)
%%
%% Dr.-Ing. Erik Burger
%% burger@kit.edu
%%
%% Version 1.1, 2014-11-21


\chapter{Appendix}   
\label{chap:appendix}


%% -------------------
%% | Example content |
%% -------------------
\section{Additional Background Plots}
\label{sec:appendix:bkgplots}
\begin{figure}
	\centering
	\begin{subfigure}{0.8\textwidth}
		\includegraphics[width=0.5\textwidth]{plots/bkg/mlvj/el_TTbar_mlvj_sb_loHPW_with_pull_log.pdf}		
		\includegraphics[width=0.5\textwidth]{plots/bkg/mlvj/mu_TTbar_mlvj_sb_loHPW_with_pull_log.pdf}		
		\caption{}
	\end{subfigure}
	\begin{subfigure}{0.8\textwidth}
		\includegraphics[width=0.5\textwidth]{plots/bkg/mlvj/el_STop_mlvj_sb_loHPW_with_pull_log.pdf}
		\includegraphics[width=0.5\textwidth]{plots/bkg/mlvj/mu_STop_mlvj_sb_loHPW_with_pull_log.pdf}	
		\caption{}	
	\end{subfigure}
	\begin{subfigure}{0.8\textwidth}
		\includegraphics[width=0.5\textwidth]{plots/bkg/mlvj/el_VV_mlvj_sb_loHPW_with_pull_log.pdf}
		\includegraphics[width=0.5\textwidth]{plots/bkg/mlvj/mu_VV_mlvj_sb_loHPW_with_pull_log.pdf}
		\caption{}
	\end{subfigure}	
	\caption[Shapes of the minor backgrounds and \ttbar \ in the \MWV -spectrum for the sideband region]{Shapes of the minor backgrounds and \ttbar \ in the \MWV -spectrum for the sideband region corresponding to $\Mpr \in [40,65]\cap[105,150]$. The shapes are taken from fits of predefined functions to the MC simulation. Shown are \ttbar \ (a), single top (b) and diboson (c) production in the electron (left) and muon channel.}
	\label{fig:app:mwv_sb_minor}
\end{figure}

\begin{figure}
	\centering
	\begin{subfigure}{0.8\textwidth}
		\includegraphics[width=0.5\textwidth]{plots/bkg/mlvj/el_TTbar_mlvj_signal_regionHPW_with_pull_log.pdf}		
		\includegraphics[width=0.5\textwidth]{plots/bkg/mlvj/mu_TTbar_mlvj_signal_regionHPW_with_pull_log.pdf}		
		\caption{}
	\end{subfigure}
	\begin{subfigure}{0.8\textwidth}
		\includegraphics[width=0.5\textwidth]{plots/bkg/mlvj/el_STop_mlvj_signal_regionHPW_with_pull_log.pdf}
		\includegraphics[width=0.5\textwidth]{plots/bkg/mlvj/mu_STop_mlvj_signal_regionHPW_with_pull_log.pdf}	
		\caption{}	
	\end{subfigure}
	\begin{subfigure}{0.8\textwidth}
		\includegraphics[width=0.5\textwidth]{plots/bkg/mlvj/el_VV_mlvj_signal_regionHPW_with_pull_log.pdf}
		\includegraphics[width=0.5\textwidth]{plots/bkg/mlvj/mu_VV_mlvj_signal_regionHPW_with_pull_log.pdf}
		\caption{}
	\end{subfigure}	
	\caption[Shapes of the minor backgrounds and \ttbar \ in the \MWV -spectrum for the signal region (WW-category)]{Shapes of the minor backgrounds and \ttbar \ in the \MWV -spectrum for the signal region (WW-category) corresponding to $\Mpr \in [65,85]$\,GeV. The shapes are taken from fits of predefined functions to the MC simulation. Shown are \ttbar \ (a), single top (b) and diboson (c) production in the electron (left) and muon channel.}
	\label{fig:app:mwv_sig_minor_WW}
\end{figure}

\begin{figure}
	\centering
	\begin{subfigure}{0.8\textwidth}
		\includegraphics[width=0.5\textwidth]{plots/bkg/mlvj/el_TTbar_mlvj_signal_regionHPZ_with_pull_log.pdf}		
		\includegraphics[width=0.5\textwidth]{plots/bkg/mlvj/mu_TTbar_mlvj_signal_regionHPZ_with_pull_log.pdf}		
		\caption{}
	\end{subfigure}
	\begin{subfigure}{0.8\textwidth}
		\includegraphics[width=0.5\textwidth]{plots/bkg/mlvj/el_STop_mlvj_signal_regionHPZ_with_pull_log.pdf}
		\includegraphics[width=0.5\textwidth]{plots/bkg/mlvj/mu_STop_mlvj_signal_regionHPZ_with_pull_log.pdf}	
		\caption{}	
	\end{subfigure}
	\begin{subfigure}{0.8\textwidth}
		\includegraphics[width=0.5\textwidth]{plots/bkg/mlvj/el_VV_mlvj_signal_regionHPZ_with_pull_log.pdf}
		\includegraphics[width=0.5\textwidth]{plots/bkg/mlvj/mu_VV_mlvj_signal_regionHPZ_with_pull_log.pdf}
		\caption{}
	\end{subfigure}	
	\caption[Shapes of the minor backgrounds and \ttbar \ in the \MWV -spectrum for the signal region (WZ-category)]{Shapes of the minor backgrounds and \ttbar \ in the \MWV -spectrum for the signal region (WZ-category) corresponding to $\Mpr \in [85,105]$\,GeV. The shapes are taken from fits of predefined functions to the MC simulation. Shown are \ttbar \ (a), single top (b) and diboson (c) production in the electron (left) and muon channel.}
	\label{fig:app:mwv_sig_minor_WZ}
\end{figure}

\begin{figure}
	\centering
	\begin{subfigure}{\textwidth}
		\centering
		\includegraphics[width=0.5\textwidth]{plots/bkg/mlvj/el_WJets_mlvj_sb_loHPW_with_pull_log.pdf}		
		\caption{}		
	\end{subfigure}
	\begin{subfigure}{\textwidth}
		\includegraphics[width=0.5\textwidth]{plots/bkg/mlvj/el_WJets_mlvj_signal_regionHPW_with_pull_log.pdf}
		\includegraphics[width=0.5\textwidth]{plots/bkg/mlvj/el_WJets_mlvj_signal_regionHPZ_with_pull_log.pdf}	
		\caption{}
	\end{subfigure}
	\begin{subfigure}{\textwidth}
		\includegraphics[width=0.5\textwidth]{plots/bkg/alpha/alpha_el_HPW.pdf}
		\includegraphics[width=0.5\textwidth]{plots/bkg/alpha/alpha_el_HPZ.pdf}
		\caption{}
	\end{subfigure}	
	\caption[Shape of W+jets production in the sideband and signal regions as well as alpha-function in the electron channel]{Shape of W+jets production in the sideband (a) and signal regions (b) as well as alpha-function (c) in the electron channel for the WW- (left) and WZ-category (right). The left axis of the alpha-function plot shows the values of the W+jets function in the signal and sideband region in arbitrary units. The right axis shows the values of the alpha-function.}
	\label{fig:app:mwvmc_alpha_el}
\end{figure}

\clearpage
\section{Additional Signal Plots}

\begin{figure}[b]
	\centering
	\begin{subfigure}{0.45\textwidth}
		\includegraphics[width=\textwidth]{plots/signal/wwwz_SM_el.pdf}
		\caption{}
	\end{subfigure}
	\begin{subfigure}{0.45\textwidth}
		\includegraphics[width=\textwidth]{plots/signal/wwwz_cwww_el.pdf}
		\caption{}
	\end{subfigure}
	\begin{subfigure}{0.45\textwidth}
		\includegraphics[width=\textwidth]{plots/signal/wwwz_ccw_el.pdf}
		\caption{}
	\end{subfigure}
	\begin{subfigure}{0.45\textwidth}
		\includegraphics[width=\textwidth]{plots/signal/wwwz_cb_el.pdf}
		\caption{}
	\end{subfigure}
	\caption[Comparison of the WW and WZ MC sample in the \Mpr -spectrum in the electron channel]{Comparison of the WW and WZ MC sample in the \Mpr -spectrum in the electron channel for different scenarios: SM (a), $\cwww=12$\,TeV$^{-2}$ (b), $\ccw=20$\,TeV$^{-2}$ (c) and $\cb=60$\,TeV$^{-2}$ (d). }
	\label{fig:signal:wwwz_comp_el}
\end{figure}

\begin{figure}
	\centering
	\begin{subfigure}{\textwidth}
		\includegraphics[width=0.5\textwidth]{plots/signal/yields_cwww_WW_el.pdf}
		\includegraphics[width=0.5\textwidth]{plots/signal/yields_cwww_WZ_el.pdf}
	\end{subfigure}
	\begin{subfigure}{\textwidth}
		\includegraphics[width=0.5\textwidth]{plots/signal/yields_ccw_WW_el.pdf}
		\includegraphics[width=0.5\textwidth]{plots/signal/yields_ccw_WZ_el.pdf}
	\end{subfigure}
	\begin{subfigure}{\textwidth}
		\includegraphics[width=0.5\textwidth]{plots/signal/yields_cb_WW_el.pdf}
		\includegraphics[width=0.5\textwidth]{plots/signal/yields_cb_WZ_el.pdf}
	\end{subfigure}
	\caption[Relative yields of the aTGC contributions in the electron channel]{Relative yields of the aTGC contributions in the electron channel.}
	\label{fig:app:atgcyields_el}
\end{figure}
		
\begin{figure}
	\centering
	\begin{subfigure}{\textwidth}
		\includegraphics[width=0.5\textwidth]{plots/signal/cwww_pos_WW_el.pdf}
		\includegraphics[width=0.5\textwidth]{plots/signal/cwww_neg_WW_el.pdf}
		\caption{}
	\end{subfigure}
	\begin{subfigure}{\textwidth}
		\includegraphics[width=0.5\textwidth]{plots/signal/ccw_pos_WW_el.pdf}
		\includegraphics[width=0.5\textwidth]{plots/signal/ccw_neg_WW_el.pdf}
		\caption{}
	\end{subfigure}
	\begin{subfigure}{\textwidth}
		\includegraphics[width=0.5\textwidth]{plots/signal/cb_pos_WW_el.pdf}
		\includegraphics[width=0.5\textwidth]{plots/signal/cb_neg_WW_el.pdf}
		\caption{}
	\end{subfigure}
	\caption[Fit result for the quadratic aTGC contribution in the WW-category, electron channel]{}
	\label{fig:signal:WW_el_sig}
\end{figure}

\begin{figure}
	\centering
	\begin{subfigure}{\textwidth}
		\includegraphics[width=0.5\textwidth]{plots/signal/cwww_pos_WZ_mu.pdf}
		\includegraphics[width=0.5\textwidth]{plots/signal/cwww_neg_WZ_mu.pdf}
		\caption{}
	\end{subfigure}
	\begin{subfigure}{\textwidth}
		\includegraphics[width=0.5\textwidth]{plots/signal/ccw_pos_WZ_mu.pdf}
		\includegraphics[width=0.5\textwidth]{plots/signal/ccw_neg_WZ_mu.pdf}
		\caption{}
	\end{subfigure}
	\begin{subfigure}{\textwidth}
		\includegraphics[width=0.5\textwidth]{plots/signal/cb_pos_WZ_mu.pdf}
		\includegraphics[width=0.5\textwidth]{plots/signal/cb_neg_WZ_mu.pdf}
		\caption{}
	\end{subfigure}
	\caption[Fit result for the quadratic aTGC contribution in the WZ-category, muon channel]{}
	\label{fig:signal:WZ_mu_sig}
\end{figure}
		
\begin{figure}
	\centering
	\begin{subfigure}{\textwidth}
		\includegraphics[width=0.5\textwidth]{plots/signal/cwww_pos_WZ_el.pdf}
		\includegraphics[width=0.5\textwidth]{plots/signal/cwww_neg_WZ_el.pdf}
		\caption{}
	\end{subfigure}
	\begin{subfigure}{\textwidth}
		\includegraphics[width=0.5\textwidth]{plots/signal/ccw_pos_WZ_el.pdf}
		\includegraphics[width=0.5\textwidth]{plots/signal/ccw_neg_WZ_el.pdf}
		\caption{}
	\end{subfigure}
	\begin{subfigure}{\textwidth}
		\includegraphics[width=0.5\textwidth]{plots/signal/cb_pos_WZ_el.pdf}
		\includegraphics[width=0.5\textwidth]{plots/signal/cb_neg_WZ_el.pdf}
		\caption{}
	\end{subfigure}
	\caption[Fit result for the quadratic aTGC contribution in the WZ-category, electron channel]{}
	\label{fig:signal:WZ_el_sig}
\end{figure}
%% ---------------------
%% | / Example content |
%% ---------------------