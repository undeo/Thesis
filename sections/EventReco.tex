\chapter{Event Reconstruction and Selection}
\label{chap::EventReconstructionandSelection}

\section{Lepton Reconstruction}
\section{Reconstruction of Missing Transverse Energy}
\section{Jet Reconstruction and Jet-Substructure Algorithms}
==Anti-k$_{\rm T}$ Algorithm
==CA-Algorithm
\subsection{Pruning}
\label{sec:pruning}
\subsection{Reconstruction of b-quark Jets}
\section{Particle Flow Algorithm}
\section{Event Simulation}
\label{sec:MC}
To be able to make reliable predictions about the events, precise simulation of the involved physical processes, like the scattering or the hadronization, as well as the detector response are needed.
\dots \\
For each aTGC parameter two working points are defined at about the expected sensitivity of the Analysis (table~\ref{tab:signal:aTGCpoints}). The MC samples for these working points are obtained by reweighting the originally created samples.
\begin{table}
	\centering
	\caption[aTGC working points]{aTGC parameter values to which the signal MC sample has been reweighted.}
	\label{tab:signal:aTGCpoints}
	\begin{tabular}{cccc}
	\hline
	\# & \Tcwww & \Tccw & \Tcb \\
	\hline
	1,2 &  $\pm 12$ & 0 & 0\\
	3,4 & 0 &  $\pm 20$ & 0\\
	5,6 & 0 & 0 &  $\pm 60$\\
	7,8 &	$\pm 12$ & $\pm 20$ & $\pm 60$\\
	\hline
	\end{tabular}
\end{table} 
=={Data to MC corrections:}
=={Pileup}
=={Reconstruction Efficiencies}
=={Jet Smearing}
\section{Control Regions}
\section{Event Selections}
