\chapter{Event Reconstruction and Selection}
\label{chap::EventReconstructionandSelection}
A crucial point of every analysis in particle physics is the reconstruction of events from the signals provided by the particle detectors. The same reconstruction has to be applied to the simulated events to be able to make reliable predictions of the background processes. By applying cuts to the variables describing each event, the signal processes can be enhanced while suppressing background events. The data used in this analysis was recorded by the CMS detector in 2015 and correponds to an integrated luminosity of 2.3\,fb$^{-1}$.\\

\noindent In the first part of this chapter, the reconstruction methods and algorithms used in this analysis are described. Some techniques that are important especially for this analysis are portrayed in more detail. The second part of the chapter describes the event selection, while the last part focusses on the event generators used for the event simulation.
\section{Reconstruction of Events}
To be able to analyse the physical processes in the particle collisions measured by the CMS detector, the signals from the different subdetectors have to be interpreted. For that the CMS collaboration uses the Particle Flow (PF) algorithm \cite{PF1,PF2}, which combines the information from different subdetectors to improve the quality of the reconstruction.
\subsection{Particle Tracks}
Particle tracks are reconstructed using the silicon tracker and the muon system described in chapter~\ref{chap:expsetup}. Charged particles propagating through the magnetic field of the solenoid move on helices, leading to bent trajectories. They leave energy deposits inside the tracker, so-called hits, which are used to reconstruct the tracks. From the curvature of the tracks the momentum as well as the charge of particles are determined. The particle tracks are also used for the identification of b quarks as described in section~\ref{sec:jetreco}.\\

\noindent The algorithm used for the reconstruction of particle tracks is the Combinatorial Track Finder (CTF) \cite{ctf1,ctf2}. First, track candidates are seeded from two or three hits close to the beam pipe in different layers of the tracker. These are then used in the Combinatorial Kalman filter method \cite{kalman} to add more hits to the track candidate. For each added hit a $\chi^2$ test is performed to see if it is compatible with the track predicted by the seeded hits. With each added hit the track prediction is updated, taking into account effects of scattering or energy losses. This fit procedure is performed parallel for many track candidates. If hits are assigned to more than one track, the tracks with lower quality, i.e. the ones with fewer compatible hits or lower $\chi^2$ values, are rejected. As a final step, the remaining track candidates are refitted to all respective hits to further improve the measurement of the track parameters.

\subsection{Vertex Reconstruction}
From the reconstructed particle tracks the collision vertices can be determined. Since there are multiple collisions in each proton bunch crossing, this is important to reduce the effects of the pileup. The vertices are reconstructed on trigger level using the information given by the silicon tracker, leading to a one-dimensional in the z-direction. A three-dimensional measurement of the vertices is obtained by using the adaptive vertex reconstruction \cite{vertexreco}. For this a $\chi^2$ is defined and minimized, taking into account a beam spot constraint, where the weights corresponding to the tracks depend on their distance to the vertex.
\subsection{Reconstruction of Electron Candidates}
Using the PF algorithm, electrons are reconstructed by combining their tracks in the silicon tracker and the energy deposited in the electromagnetic calorimeter with a Gaussian sum filter \cite{elreco1,elreco2}. Being deflected in the magnetic field of the detector, the electrons loose a significant amount of energy through bremsstrahlung. The photons generated by this bremsstrahlung are emitted tangential to the electron track. This causes a spread of the energy deposited in the electromagnetic calorimeter the in $\phi$-direction, which is taken into account by the reconstruction algorithm by forming so-called super-clusters. For electrons with high $p_{\rm T}$ the reconstruction starts by searching for such super-clusters in the electromagnetic calorimeter, while electrons with low $p_{\rm T}$ are reconstructed beginning with their track.\\

\noindent Electron candidates have to pass several identification criteria optimized for high-$p_{\rm T}$ electrons \cite{elreq}. Additionally they are required to be isolated, i.e. no other particle track must lie within $\Delta=0.3$ of the reconstructed electron.
\subsection{Reconstruction of Muon Candidates}
Due to the high mass of muons compared to electrons, the amount of energy lost through bremsstrahlung that is deposited in the electromagnetic calorimeter is negligible. Therefore, they can propagate through the whole detector and leave a trace in the muon chambers. Muon candidates are reconstructed by combining their track in the silicon tracker and the entries in the muon system.  
\subsection{Missing Transverse Energy}
The neutrino produced in the semileptonic decay of the W boson leaves the detector without any interaction. However, some of its properties can be reconstructed using the laws of momentum and energy conservation. The two quarks participating in the collision are expected to have negligible transverse momentum. Therefore, the sum of the transverse momenta of all final state particles is expected to be zero. Any non-zero net momentum of the final state particle is assigned to the neutrino\footnote{In the relativistic limit the energy $E$ is equal to the momentum $p$ (assuming natural units, $c=\hbar=1$)}\cite{MET}
\begin{equation}
\MET \equiv -\sum_k p_{{\rm T},k} ~,
\end{equation}
where the sum is over all reconstructed particles. Since there is only one neutrino in each event, this \MET can be directly assigned to it. From the \MET together with the W mass constraint the longitudinal momentum of the neutrino is reconstructed \cite{PAS} via
\begin{align}
p_{z,\nu} &= \frac{\mu \cdot p_{z,{\rm lep}}}{p_{\rm T,lep}^2} \pm \sqrt{\frac{\mu^2 \cdot p_{z,{\rm lep}}^2}{p_{\rm T,lep}^4} - \frac{E_{\rm lep}^2 \cdot \MET^2 - \mu}{p_{\rm T,lep}^2}} ~, \\
\mu &= \frac{M_W^2}{2} + \vec{p}_{\rm T,lep}\cdot \vec{\MET} ~,
\end{align} 
where $p_{z,{\rm lep}}$ and $p_{\rm T,lep}$ are the longitudinal  and transverse momentum of the lepton, respectively, and $E_{\rm lep}$ is its total energy. In case of a imaginary solution only the real part is used, while for two solutions the smaller one is used.
\subsection{Reconstruction of Photons and Neutral Hadrons}
Neutral particles like photons and neutral hadrons do not leave any trace in the silicon tracker. Therefore, they can only be reconstructed using the calorimeters, leading to higher uncertainties on the reconstructed kinematic variables. Depending on wether a energy deposit without associated track is found in the electromagnetic or the hadronic calorimeter, it is identified as photon or neutral hadron, respectively. 
\subsection{Jet Reconstruction and Jet-Substructure Algorithms}
\label{sec:jetreco}
Quarks and gluons in the final state can not be directly detected since they are colour-charged. Due to gluon radiation and the splitting of gluons in quark-antiquark pairs, a shower of particle develops, a so-called jet. A jet is reconstructed using the entries in the hadronic calorimeter. There are different algorithms available for the reconstruction of jets, which can be divided into two categories. Cone-like algorithms fit cones with fixed radius to the calorimeter entries centred at the hardest particle (e.g. the $k_{\rm T}$-algorithm), while sequential algorithms combine particles depending on a distance parameter $d$ (e.g. anti-$k_{\rm T}$). Latter algorithms are used in this analysis, which is why they are described in more detail here.\\

\noindent For sequential jet reconstruction algorithms the distance between two particles $d_{ij}$ is defined as
\begin{align}
d_{ij} = \min\left(k_{{\rm T},i}^{2p},k_{{\rm T},j}^{2p}\right)\frac{\Delta_{ij}^2}{R^2} ~,
\end{align}
with
\begin{equation}
\Delta_{ij}^2 = (y_i-y_j)^2 + (\phi_i - \phi_j)^2 ~, \label{eq:reco:deltar}
\end{equation}
where $R$ denotes a fixed size parameter, $k_{{\rm T},i}$ the transverse momentum, $y_i$ the rapidity and $\phi_i$ the azimuthal angle of particle $i$. The value of the parameter $p$ depends on the algorithm used and is crucial for the properties of the reconstruction. The distance of a particle and the beam $d_{iB}$ is given by
\begin{equation}
d_{iB} = k_{{\rm T},i}^{2p} ~.
\end{equation}
These distance parameters are estimated for all particle combinations. The two particles with the smallest $d_{ij}$ are then combined to a proto-jet, which is treated equally as particle in the next iterations. If the smallest distance happens to be a $d_{iB}$, the proto-jet $i$ is promoted to a jet and removed from the list of particles. This procedure is repeated until all particles have been assigned to a jet.\\

\noindent For the Cambridge-Aachen algorithm (CA) $p=-1$ is used \cite{CAalgo}. For this algorithm the distances $d$ only depend on the spatial separation of the particles. This makes it useful for jet substructure methods, like the pruning algorithm described below.\\

\noindent A value of $p=-1$ corresponds to the anti-$k_{\rm T}$ algorithm. All soft particles within a radius of $R$ around a hard particle are clustered with the hard particle before clustering among themselves. If the distance between two hard particles is smaller than $R$, the particle with higher $k_{\rm T}$ is prioritized. This leads to round, cone-like jets, where soft particles do not affect the jet shapes.\\

\noindent A comparison of the jet reconstruction of a simulated event with many soft particles is shown in Fig.~\ref{fig:eventreco:jet_reco}. As can be seen there, the anti-$k_{\rm T}$ produces cone-like jets that are circular in the $y$-$\phi$-plane. They are centred around the particles with high $p_{\rm T}$.\\
\begin{figure}
	\centering
	\subfigure[]{
		\includegraphics[width=0.45\textwidth]{../plots/reco/jet_reco_antikt.png}
	}
	\subfigure[]{
		\includegraphics[width=0.45\textwidth]{../plots/reco/jet_reco_CA.png}
	}
	\caption[Comparison of the anti-$k_{\rm T}$ and CA jet reconstruction algorithms]{Comparison of the anti-$k_{\rm T}$ (a) and CA (b) jet reconstruction algorithms in the $p_{\rm T}$ distribution in the $y$-$\phi$-plane \cite{antikt}. The anti-$k_{\rm T}$ algorithm results in circular jets in the $y$-$\phi$-plane centred around the hardest particles, which are indicated by the high $p_{\rm T}$. On the other hand the CA algorithm produces jets with irregular shapes that are strongly influenced by soft particles.}
	\label{fig:eventreco:jet_reco}
\end{figure}

\noindent The momentum of the jet candidates is estimated as the vectorial sum of all particle momenta in the jet. For simulated events the resulting momenta are found to be within 5 to 10\,\% of the true particle momenta. Due to particles from pileup events that are falsely reconstructed as part of the jets an offset correction is applied. The energy correction are validated with in situ measurements in dijet and photon + jet events \cite{jetcorrection}.\\

\noindent A high $p_{\rm T}$ cut is applied to the hadronically decaying boson (see section~\ref{sec:eventselection}). The jets emerging from the two quarks in the final state are therefore overlapping. Instead of two distinct jets, only one so-called fat jet can be reconstructed. Therefore, two size parameters $R=0.4$ and $R=0.8$ are used for the anti-$k_{\rm T}$ algorithm, corresponding to AK4 and AK8 jets, respectively. The AK8 jet with highest transverse momentum is used as hadronic boson candidate while the AK4 jets are used to identify b quark jets. The substructure of the AK8 jet can be used to discriminate jets coming from QCD events. In this analysis, the $N$-subjettiness and the pruning algorithm are used, which are also described in this section.

\subsubsection*{Reconstruction of b Quark Jets}
Jets resulting from b quarks can be differed from other jets. Due to the colour confinement, b quarks tend to form B mesons. Since their decay into the lighter c and d quarks is suppressed by the CKM mechanism \cite{CKM}, they have a high mean life time of $\tau=1,6\cdot 10^{-12}$\,s \cite{SMmasses}. Before decaying and forming a jet, they travel a measurable distance away from the primary vertex, forming a secondary vertex (see Fig.~\ref{fig:reco:bjet}). In this analysis, the Combined Secondary Vertex (CSV) algorithm \cite{CSV}\textbf{?CHECK?} is used to identify b quark jets. It makes use of different kinematic variables related to the secondary vertex, e.g. the invariant mass of charged particles or their rapidity as well as the impact parameters of tracks coming from the secondary vertex to construct a discriminating variable. Working points are defined as different cuts on the discriminating variable, resulting in different reconstruction efficiencies. In this analysis, the tight working point is used. Events are excluded from the analysis if they contain one or more AK4 jets with $p_{\rm T}>30$\,GeV and $|\eta|<2.4$ that are identified as b quark jets. Additionally, the AK4 jet has to be in a distance of $\Delta R \geq 0.8$ with respect to the hadronic boson candidate to avoid the exclusion of WZ events with the Z decaying to b$\bar{\rm b}$.
\begin{figure}
    \centering
    \includegraphics[width=0.5\textwidth]{../plots/reco/bjet.png}
    \caption[Illustration of a b quark jet]{Illustration of a b quark jet, taken from \cite{bjet}. The tracks of one of the jets do not point to the primary vertex but rather to a secondary one displaced by the impact parameter $d_0$.}\
    \label{fig:reco:bjet}
\end{figure}

\subsubsection*{$N$-Subjettiness}
Particle jets arising from QCD events like W+jets production are expected to have a different substructure compared to jets from hadronically decaying boosted bosons. A variable with discrimination power between those jets is the $N$-subjettiness \cite{nsubjettiness}. First, the jet is reclustered to $N$ subjets using a reconstruction algorithm. For $N$ reconstructed subjets the $N$-subjettiness is defined as
\begin{align}
\tau_N &= \frac{1}{d_0} \sum_k p_{{\rm T},k} \min(\Delta_{1k},\Delta_{2k},\dots,\Delta_{Nk}) ~, \\
d_0 &= \sum_k p_{{\rm T},k} R_0 ~,
\end{align}
where $k$ runs over all constituent particles of a given jet, $p_{{\rm T},k}$ is the transverse momentum of jet $k$, $R_0$ is the jet radius used in the original reconstructions algorithm and the $\Delta_{ik}$ are defined in equation~\ref{eq:reco:deltar}. $\tau_N$ quantifies to what degree the jet can be regarded to as composed of $N$ subjets. Rather than applying a cut directly on $\tau_N$, the 2-1-subjettiness, which is defined as the ratio
\begin{equation}
\tau_{21}=\frac{\tau_2}{\tau_1} ~,
\end{equation}
is found to be a much more discriminating variable between heavy boson jets and jets emerging from gluons or single quarks. In Fig.~\ref{fig:reco:nsubjet} exemplary jet substructures of a boosted QCD jet and a jet emerging from the decay of a boosted heavy boson are shown. Since it consists of two subjets, the 2-1-subjettiness yields a lower value for the latter one.
\begin{figure}
    \centering
    \subfigure[]{
	    \includegraphics[width=0.45\textwidth]{../plots/reco/eventDisplayQCDW.pdf}
	}
	\subfigure[]{
	    \includegraphics[width=0.45\textwidth]{../plots/reco/eventDisplayW.pdf}
	}
    \caption[Comparison of a boosted QCD jet and a jet emerging from the decay of a boosted W boson in the $y$-$\phi$-plane]{Comparison of a boosted QCD jet (a) and a jet emerging from the decay of a boosted W boson in the $\eta$-$\phi$-plane. The dashed lines indicate the borders of the reconstructed jets. The jet from the W boson decay shows two spatially separated subjets.}
    \label{fig:reco:nsubjet}
\end{figure}
\subsubsection*{Pruning}
The so-called pruning algorithm is another way of discriminating jets from QCD events by removing soft and wide-angle radiation that emerges from e.g. the underlying event  \cite{pruning}. The basic idea is to rerun the CA algorithm on the AK8 jets reconstructed with the anti-$k_{\rm T}$ algorithm. Since the CA algorithm is used, only the spatial substructure is taken into account.  Before recombining particle $i$ and $j$ to a subjet $p$, the following conditions have to be met
\begin{align}
z = \frac{\min (p_{{\rm T}i},p_{{\rm T}j})}{p_{{\rm T}p}} &> z_{\rm cut} ~, \\
\Delta_{ij} &< D_{\rm cut} = \frac{2r_{\rm cut}M_j}{p_{\rm T}} ~,
\end{align}
where $M_j$ denotes the mass and $p_{\rm T}$ the transverse momentum of the originally reconstructed jet and $p_{{\rm T}}p$ is the transverse momentum of the resulting subjet. If this is not the case, the merging is vetoed and the softer of the two particles is removed. The resulting pruned jet mass \Mpr \ is much more robust against systematic effects resulting from the underlying event. In this analysis the parameter values $z_{\rm cut}=0.1$ and $r_{\rm cut}=0.5$ are used.
\begin{figure}
	\centering
	\subfigure[]{
		\includegraphics[width=0.45\textwidth]{../plots/reco/ttbar_noUE.pdf}
	}
	\subfigure[]{
		\includegraphics[width=0.45\textwidth]{../plots/reco/ttbar_UE.pdf}
	}
	\caption[Comparison of the pruned and unpruned jet masses for simulated top quark decays with and without the underlying event]{Comparison of the unpruned (a) and pruned jet masses (b) for simulated top quark decays with (straight line) and without the underlying event (dashed line). The unpruned jet mass gets shifted to higher masses, while for the pruned jet mass the underlying event hardly affects the resulting distribution.}
	\label{fig:reco:pruning}
\end{figure}
\subsection{Reconstruction of the Diboson System}
The main observable in this analysis is the invariant mass of the two bosons, $M_{WV}$. It is obtained from the sum of the four-vectors of the hadronic and leptonic bosons. The hadronic boson candidate is taken as the AK8 jet with highest $p_{\rm T}$. The leptonic W candidate is reconstructed as the sum of the four-vectors of the lepton and neutrino. 

\section{Event Selections}
\label{sec:eventselection}
Each event has to pass several selection criteria, which are applied in form of cuts on event variables. These cuts are specifically chosen to reduce the background contributions while enhancing the signal process. To reduce noise of the subdetectors, at least one well reconstructed collision vertex is required for each event.\\

\noindent Since the effects of anomalous triple gauge coupling are expected to be most visible in the high-$p_{\rm T}$-regime, a cut of $p_{\rm T}>200$\,GeV and $|\eta|<2.4$ on both the hadronic and the leptonic boson candidate is applied. The signal process contains one leptonically decaying W boson, so each event is required to contain exactly one electron (muon) with a transverse momentum of $p_{\rm T}>50$\,GeV (53\,GeV) that lies within $|\eta|<2.5$ (2.4), while for the missing transverse energy a cut of $\MET >80$\,GeV (40\,GeV) is applied. This lepton is used to split the data into an electron and a muon channel. An event is vetoed if there is an additional lepton with lower $p_{\rm T}$ requirements ($p_{\rm T}>20$\,GeV).  For jets originating from boosted heavy bosons a low value of the $N$-subjettiness $\tau_{21}$ is expected, which is why a cut of $\tau_{21}<0.6$ is applied.\\

\noindent The pruned jet mass reconstructed from the hadronic W or Z boson decays are exoected to peak at the W or Z boson mass, respectively. Due to the limited energy resolution of the hadronic calorimeter the peaks are smeared out. A cut of $40\,{\rm GeV}</Mpr<150\,{\rm GeV}$ is applied while the regions with $\Mpr<60$\,GeV and $\Mpr>105$\, GeV are considered to be dominated by background processes and are used for the background estimation as explained in chapter~\ref{chap:bkg}. A signal region used in the final limit extraction is defined as $65\,{\rm GeV}<\Mpr<105$\,GeV. The main observable in this analysis is the invariant mass of the diboson system $\MWV$ on which a cut of $\MWV>900$\,GeV is applied to avoid possible turn on effects in the background contributions at low invariant masses.

\section{Event Simulation}
\label{sec:MC}
To be able to make reliable predictions about the events, precise simulation of the involved physical processes, like the scattering or the hadronization, as well as the detector response are needed. For this, different Monte Carlo event generators are used for the different processes relevant in this analysis. The event simulation can be seperated into several parts as shown in Fig.~\ref{fig:reco:simparts}. Starting from the parton distribution functions, which describe the distribution of enery between the partons of the colliding protons, the hard scattering process is calculated. Then, the parton shower and the hadroniyation, including the decay of the final state partices, is simulated. In this section, the different event generators are shortly described. More detailed information can be found in the cited documents.\\
\begin{figure}
    \centering
    \includegraphics[width=0.75\textwidth]{../plots/reco/simparts.pdf}
    \caption[Schemaric overview of the different steps of the event generation]{Schemaric overview of the different steps of the event generation, taken from \cite{MCfig}}
    \label{fig:reco:simparts}
\end{figure}

\subsection*{\textsc{MadGraph}}
\textsc{MadGraph} \cite{madgraph} is a Python-based event generator capable of simulating all SM processes as well as any arbitrary user-defined process at leading order. Tree-level amplitudes are calculated by generating all relevant Feynman diagrams and the phase space of a given process. Corresponding events and cross sections can then be generated using the \textsc{MadEvent} package. Parton showering as well as hadronization are not implemented and have to be performed using \textsc{Pythia}, which is described below.
\subsection*{\textsc{MadGraph\textunderscore aMC@NLO}}
The \textsc{MadGraph\textunderscore aMC@NLO} event generator is a combination of the \textsc{MadGraph} and \textsc{MC@NLO} event generators that allows for next to leading order (NLO) simulations \cite{madgraphamc}. It can also be used as parton shower generator. In this case the double-counting of events that are added both by the NLO and the parton shower algorithm has to be taken into account. This can lead to negative event weights.
\subsection*{\textsc{Powheg}}
The Positive Weight Hardest Emission Generator or \textsc{Powheg} is a matrix element generator at NLO. Since 
\subsection*{\textsc{Pythia}}
\textsc{Pythia} \cite{pythia} is a mulit-purpose event generator that includes parton showering and hadronization based on the Lund string model. Since it only supports LO calculations for the hard scattering processes it is mostly used in combination with \textsc{Madgraph} or \textsc{Powheg} only providing the parton shower and hadronization.
\subsection*{\textsc{Geant4}}
To be able to compare the simulated events with the measured data, the particle interaction with the detector material has to be simulated. This is done using \textsc{Geant4} \cite{geant4}, which is a toolkit capable of simulating particles propagating through matter. Effects like scattering, the emission of bremsstrahlung or ionization are included. 

=={Data to MC corrections:}
=={Pileup}
=={Reconstruction Efficiencies}
=={Jet Smearing}\\
\noindent The simulated samples for the background contributions are verified by comparing them with the data in two control regions. For the \ttbar \ control region the same cuts as stated in section~\ref{sec:eventselection} are applied, but at least one AK4 jet has to be identified as b quark jet. The agreement of simulation and data is verified using several observables. The resulting plots for the two observables used in this analysis (\Mpr \ and \MWV) are shown in Fig.~\ref{fig:reco:control}, while all other plots can be found in \cite{PAS}.\\

\noindent The W+jets control region (refered to as sideband region in the background estimation in chapter~\ref{chap:bkg}) is defined in the pruned jet mass spectrum as 40\,GeV$<\Mpr<$65\,GeV and 105\,GeV$<\Mpr<$150\,GeV, while all other cuts are applied as mentioned above. This region is signal depleted and dominated by the W+jets production. Again, a good agreement is found in all observables.

\begin{figure}
	\centering
	\subfigure[]{
		\includegraphics[width=0.45\textwidth]{../plots/reco/Mjpruned_el.pdf}
	}
	\subfigure[]{
		\includegraphics[width=0.45\textwidth]{../plots/reco/MWW_el.pdf}
	}
	\subfigure[]{
		\includegraphics[width=0.45\textwidth]{../plots/reco/Mjpruned_mu.pdf}
	}
	\subfigure[]{
		\includegraphics[width=0.45\textwidth]{../plots/reco/MWW_mu.pdf}
	}
	\caption[Comparison of the simulated samples and the data in the \ttbar \ control region]{Comparison of the simulated samples and the data in the \ttbar \ control region in the electron (a,b) and muon channel (c,d), taken from \cite{PAS}. Shown are the \Mpr \ distributions (a,c) and the \MWV \ distributions (b,d). A good agreement between simulation and data is found in all channels.}
	\label{fig:reco:control}
\end{figure}
