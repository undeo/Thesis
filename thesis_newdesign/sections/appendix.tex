%% LaTeX2e class for student theses
%% sections/apendix.tex
%% 
%% Karlsruhe Institute of Technology
%% Institute for Program Structures and Data Organization
%% Chair for Software Design and Quality (SDQ)
%%
%% Dr.-Ing. Erik Burger
%% burger@kit.edu
%%
%% Version 1.1, 2014-11-21


\chapter{Appendix}   
\label{chap:appendix}


%% -------------------
%% | Example content |
%% -------------------

\section{Additional Signal Figures}

\begin{figure}[h!]
	\centering
	\subfigure[]{
		\includegraphics[width=0.5\textwidth]{../plots/signal/wwwz_SM_el.pdf}
		
	}%
	\subfigure[]{
		\includegraphics[width=0.5\textwidth]{../plots/signal/wwwz_cwww_el.pdf}
		
	}
	\subfigure[]{
		\includegraphics[width=0.5\textwidth]{../plots/signal/wwwz_ccw_el.pdf}
		
	}%
	\subfigure[]{
		\includegraphics[width=0.5\textwidth]{../plots/signal/wwwz_cb_el.pdf}
		
	}
	\caption[Comparison of the WW and WZ sample in the \Mpr -spectrum in the electron channel]{Comparison of the WW and WZ sample in the \Mpr -spectrum in the electron channel for different scenarios: SM (a), $\cwww=12$\,TeV$^{-2}$ (b), $\ccw=20$\,TeV$^{-2}$ (c) and $\cb=60$\,TeV$^{-2}$ (d). }
	\label{fig:signal:wwwz_comp_el}
\end{figure}

\begin{figure}
	\centering
	\subfigure[]{
		\includegraphics[width=0.5\textwidth]{../plots/signal/yields_cwww_WW_el.pdf}
		\includegraphics[width=0.5\textwidth]{../plots/signal/yields_cwww_WZ_el.pdf}
	}
	\subfigure[]{
		\includegraphics[width=0.5\textwidth]{../plots/signal/yields_ccw_WW_el.pdf}
		\includegraphics[width=0.5\textwidth]{../plots/signal/yields_ccw_WZ_el.pdf}
	}
	\subfigure[]{
		\includegraphics[width=0.5\textwidth]{../plots/signal/yields_cb_WW_el.pdf}
		\includegraphics[width=0.5\textwidth]{../plots/signal/yields_cb_WZ_el.pdf}
	}
	\caption[Relative yields of the aTGC contributions in the electron channel]{Relative yields of the aTGC contributions in the electron channel.}
	\label{fig:app:atgcyields_el}
\end{figure}
		

\begin{figure}
	\centering
	\subfigure[]{
		\includegraphics[width=0.5\textwidth]{../plots/signal/SM_int_ccw_WW_el.pdf}
	}%
	\subfigure[]{
		\includegraphics[width=0.5\textwidth]{../plots/signal/SM_int_ccw_WW_mu.pdf}
	}	
	\subfigure[]{
		\includegraphics[width=0.5\textwidth]{../plots/signal/SM_int_ccw_WZ_el.pdf}
	}%
	\subfigure[]{
		\includegraphics[width=0.5\textwidth]{../plots/signal/SM_int_ccw_WZ_mu.pdf}
	}	
	\caption[Fit results for the interference with the SM in the WW- and WZ-category for \Tccw \ , electron and muon channel]{Fit results for the interference with the SM in the WW- (a,c) and WZ-category (b,c) for \Tccw \ in the electron (a,c) and muon channel (b,d). Shown is the difference of the simulated samples for the positive and negative working points as well as the fitted function (red line).}
\end{figure}	

\begin{figure}
	\centering
	\subfigure[]{
		\includegraphics[width=0.5\textwidth]{../plots/signal/SM_int_cb_WW_el.pdf}
	}%
	\subfigure[]{
		\includegraphics[width=0.5\textwidth]{../plots/signal/SM_int_cb_WW_mu.pdf}
	}	
	\subfigure[]{
		\includegraphics[width=0.5\textwidth]{../plots/signal/SM_int_cb_WZ_el.pdf}
	}%
	\subfigure[]{
		\includegraphics[width=0.5\textwidth]{../plots/signal/SM_int_cb_WZ_mu.pdf}
	}	
	\caption[Fit results for the interference with the SM in the WW- and WZ-category for \Tcb \ , electron and muon channel]{Fit results for the interference with the SM in the WW- (a,c) and WZ-category (b,c) for \Tcb \ in the electron (a,c) and muon channel (b,d). Shown is the difference of the simulated samples for the positive and negative working points as well as the fitted function (red line).}
\end{figure}	


\newpage


\begin{figure}
	\centering
	\subfigure[]{
		\includegraphics[width=0.5\textwidth]{../plots/signal/cwww_pos_WW_mu.pdf}
		\includegraphics[width=0.5\textwidth]{../plots/signal/cwww_neg_WW_mu.pdf}
	}
	\subfigure[]{
		\includegraphics[width=0.5\textwidth]{../plots/signal/ccw_pos_WW_mu.pdf}
		\includegraphics[width=0.5\textwidth]{../plots/signal/ccw_neg_WW_mu.pdf}
	}
	\subfigure[]{
		\includegraphics[width=0.5\textwidth]{../plots/signal/cb_pos_WW_mu.pdf}
		\includegraphics[width=0.5\textwidth]{../plots/signal/cb_neg_WW_mu.pdf}
	}
	\caption[Fit result for the quadratic aTGC contribution in the WW-category, muon channel]{Fit result for the quadratic aTGC contribution in the WW-category, muon channel, for positive(left) and negative parameter values (right), corresponding to \Tcwww \ (a), \Tccw \ (b) and \Tcb \ (c).}
\end{figure}
		
\begin{figure}
	\centering
	\subfigure[]{
		\includegraphics[width=0.5\textwidth]{../plots/signal/cwww_pos_WW_el.pdf}
		\includegraphics[width=0.5\textwidth]{../plots/signal/cwww_neg_WW_el.pdf}
	}
	\subfigure[]{
		\includegraphics[width=0.5\textwidth]{../plots/signal/ccw_pos_WW_el.pdf}
		\includegraphics[width=0.5\textwidth]{../plots/signal/ccw_neg_WW_el.pdf}
	}
	\subfigure[]{
		\includegraphics[width=0.5\textwidth]{../plots/signal/cb_pos_WW_el.pdf}
		\includegraphics[width=0.5\textwidth]{../plots/signal/cb_neg_WW_el.pdf}
	}
	\caption[Fit result for the quadratic aTGC contribution in the WW-category, electron channel]{Fit result for the quadratic aTGC contribution in the WW-category, electron channel, for positive(left) and negative parameter values (right), corresponding to \Tcwww \ (a), \Tccw \ (b) and \Tcb \ (c).}
\end{figure}

\begin{figure}
	\centering
	\subfigure[]{
		\includegraphics[width=0.5\textwidth]{../plots/signal/cwww_pos_WZ_mu.pdf}
		\includegraphics[width=0.5\textwidth]{../plots/signal/cwww_neg_WZ_mu.pdf}
		
	}
	\subfigure[]{
		\includegraphics[width=0.5\textwidth]{../plots/signal/ccw_pos_WZ_mu.pdf}
		\includegraphics[width=0.5\textwidth]{../plots/signal/ccw_neg_WZ_mu.pdf}
		
	}
	\subfigure[]{
		\includegraphics[width=0.5\textwidth]{../plots/signal/cb_pos_WZ_mu.pdf}
		\includegraphics[width=0.5\textwidth]{../plots/signal/cb_neg_WZ_mu.pdf}
		
	}
	\caption[Fit result for the quadratic aTGC contribution in the WZ-category, muon channel]{Fit result for the quadratic aTGC contribution in the WZ-category, muon channel, for positive(left) and negative parameter values (right), corresponding to \Tcwww \ (a), \Tccw \ (b) and \Tcb \ (c)..}
\end{figure}
		
\begin{figure}
	\centering
	\subfigure[]{
		\includegraphics[width=0.5\textwidth]{../plots/signal/cwww_pos_WZ_el.pdf}
		\includegraphics[width=0.5\textwidth]{../plots/signal/cwww_neg_WZ_el.pdf}
		
	}
	\subfigure[]{
		\includegraphics[width=0.5\textwidth]{../plots/signal/ccw_pos_WZ_el.pdf}
		\includegraphics[width=0.5\textwidth]{../plots/signal/ccw_neg_WZ_el.pdf}
		
	}
	\subfigure[]{
		\includegraphics[width=0.5\textwidth]{../plots/signal/cb_pos_WZ_el.pdf}
		\includegraphics[width=0.5\textwidth]{../plots/signal/cb_neg_WZ_el.pdf}
		
	}
	\caption[Fit result for the quadratic aTGC contribution in the WZ-category, electron channel]{Fit result for the quadratic aTGC contribution in the WZ-category, electron channel, for positive(left) and negative parameter values (right), corresponding to \Tcwww \ (a), \Tccw \ (b) and \Tcb \ (c)..}
\end{figure}





\newpage
\section{Additional Background Figures}
\label{sec:appendix:bkgplots}
\begin{figure}[h!]
	\centering
	\subfigure[]{
		\includegraphics[width=0.5\textwidth]{../plots/bkg/mlvj/el_TTbar_mlvj_sb_loHPW_with_pull_log.pdf}		
		\includegraphics[width=0.5\textwidth]{../plots/bkg/mlvj/mu_TTbar_mlvj_sb_loHPW_with_pull_log.pdf}		
	}
	\subfigure[]{
		\includegraphics[width=0.5\textwidth]{../plots/bkg/mlvj/el_STop_mlvj_sb_loHPW_with_pull_log.pdf}
		\includegraphics[width=0.5\textwidth]{../plots/bkg/mlvj/mu_STop_mlvj_sb_loHPW_with_pull_log.pdf}				
	}
	\subfigure[]{
		\includegraphics[width=0.5\textwidth]{../plots/bkg/mlvj/el_VV_mlvj_sb_loHPW_with_pull_log.pdf}
		\includegraphics[width=0.5\textwidth]{../plots/bkg/mlvj/mu_VV_mlvj_sb_loHPW_with_pull_log.pdf}		
	}	
	\caption[Shapes of the minor backgrounds and \ttbar \ in the \MWV -spectrum for the sideband region]{Shapes of the minor backgrounds and \ttbar \ in the \MWV -spectrum for the sideband region corresponding to $\Mpr \in [40,65]\cap[105,150]$. Shown are \ttbar \ (a), single top (b) and diboson (c) production in the electron (left) and muon channel (right).}
	\label{fig:app:mwv_sb_minor}
\end{figure}

\begin{figure}
	\centering
	\subfigure[]{
		\includegraphics[width=0.5\textwidth]{../plots/bkg/mlvj/el_TTbar_mlvj_signal_regionHPW_with_pull_log.pdf}		
		\includegraphics[width=0.5\textwidth]{../plots/bkg/mlvj/mu_TTbar_mlvj_signal_regionHPW_with_pull_log.pdf}				
	}
	\subfigure[]{
		\includegraphics[width=0.5\textwidth]{../plots/bkg/mlvj/el_STop_mlvj_signal_regionHPW_with_pull_log.pdf}
		\includegraphics[width=0.5\textwidth]{../plots/bkg/mlvj/mu_STop_mlvj_signal_regionHPW_with_pull_log.pdf}				
	}
	\subfigure[]{
		\includegraphics[width=0.5\textwidth]{../plots/bkg/mlvj/el_VV_mlvj_signal_regionHPW_with_pull_log.pdf}
		\includegraphics[width=0.5\textwidth]{../plots/bkg/mlvj/mu_VV_mlvj_signal_regionHPW_with_pull_log.pdf}		
	}	
	\caption[Shapes of the minor backgrounds and \ttbar \ in the \MWV -spectrum for the signal region (WW-category)]{Shapes of the minor backgrounds and \ttbar \ in the \MWV -spectrum for the signal region (WW-category) corresponding to $\Mpr \in [65,85]$\,GeV. Shown are \ttbar \ (a), single top (b) and diboson (c) production in the electron (left) and muon channel (right).}
	\label{fig:app:mwv_sig_minor_WW}
\end{figure}

\begin{figure}
	\centering
	\subfigure[]{
		\includegraphics[width=0.5\textwidth]{../plots/bkg/mlvj/el_TTbar_mlvj_signal_regionHPZ_with_pull_log.pdf}		
		\includegraphics[width=0.5\textwidth]{../plots/bkg/mlvj/mu_TTbar_mlvj_signal_regionHPZ_with_pull_log.pdf}		
		
	}
	\subfigure[]{
		\includegraphics[width=0.5\textwidth]{../plots/bkg/mlvj/el_STop_mlvj_signal_regionHPZ_with_pull_log.pdf}
		\includegraphics[width=0.5\textwidth]{../plots/bkg/mlvj/mu_STop_mlvj_signal_regionHPZ_with_pull_log.pdf}	
			
	}
	\subfigure[]{
		\includegraphics[width=0.5\textwidth]{../plots/bkg/mlvj/el_VV_mlvj_signal_regionHPZ_with_pull_log.pdf}
		\includegraphics[width=0.5\textwidth]{../plots/bkg/mlvj/mu_VV_mlvj_signal_regionHPZ_with_pull_log.pdf}
		
	}	
	\caption[Shapes of the minor backgrounds and \ttbar \ in the \MWV -spectrum for the signal region (WZ-category)]{Shapes of the minor backgrounds and \ttbar \ in the \MWV -spectrum for the signal region (WZ-category) corresponding to $\Mpr \in [85,105]$\,GeV. Shown are \ttbar \ (a), single top (b) and diboson (c) production in the electron (left) and muon channel (right).}
	\label{fig:app:mwv_sig_minor_WZ}
\end{figure}

\begin{figure}
	\centering
	\subfigure[]{
		\centering
		\includegraphics[width=0.5\textwidth]{../plots/bkg/mlvj/el_WJets_mlvj_sb_loHPW_with_pull_log.pdf}		
				
	}
	\subfigure[]{
		\includegraphics[width=0.5\textwidth]{../plots/bkg/mlvj/el_WJets_mlvj_signal_regionHPW_with_pull_log.pdf}
		\includegraphics[width=0.5\textwidth]{../plots/bkg/mlvj/el_WJets_mlvj_signal_regionHPZ_with_pull_log.pdf}	
		
	}
	\subfigure[]{
		\includegraphics[width=0.5\textwidth]{../plots/bkg/alpha/alpha_el_HPW.pdf}
		\includegraphics[width=0.5\textwidth]{../plots/bkg/alpha/alpha_el_HPZ.pdf}
		
	}	
	\caption[Shape of W+jets production in the sideband and signal regions as well as alpha-function in the electron channel]{Shape of W+jets production in the sideband (a) and signal regions (b) as well as alpha-function (c) in the electron channel for the WW- (left) and WZ-category (right). The left axis of the alpha-function plot shows the values of the W+jets function in the signal and sideband region in arbitrary units. The right axis shows the values of the alpha-function.}
	\label{fig:app:mwvmc_alpha_el}
\end{figure}


\section{Additional Limit Figures}

\begin{figure}
	\centering
	\subfigure[]{
		\includegraphics[width=0.6\textwidth]{../plots/results/1dlimit_lZ.pdf}
	}
	\subfigure[]{
		\includegraphics[width=0.6\textwidth]{../plots/results/1dlimit_dkz.pdf}
	}
	\subfigure[]{
		\includegraphics[width=0.6\textwidth]{../plots/results/1dlimit_dg1z.pdf}
	}		
	\caption[Delta-log-likelihood distributions for the three aTGC parameters in vertex parametrization]{$\Delta NNL$ distributions for $\lambda_Z$ \ (a), $\Delta \kappa_Z$ \ (b) and $\Delta g_1^Z$ \ (c). Shown are the expected (blue) and observed distributions (black). The intersection of the black horizontal dashed line and the expected or observed distribution corresponds to the 95\% C.L., indicated by the vertical blue and black dashed lines for the expected and observed limit, respectively. Also shown is the $\pm 1\sigma$-environment around the expected limit (shaded area).}
\end{figure}

%% ---------------------
%% | / Example content |
%% ---------------------