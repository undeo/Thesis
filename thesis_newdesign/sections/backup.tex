\begin{figure}
	\centering
	\subfigure[]{
		\includegraphics[width=0.1\textwidth]{plots/bkg/feynmans/SMWW.pdf}
		\caption{}
		\label{fig:bkg:fy_WWtgc}
	}
	\subfigure[]{
		\includegraphics[width=0.1\textwidth]{plots/bkg/feynmans/SMWZ.pdf}
		\caption{}
		\label{fig:bkg:fy_WZtgc}
	}
	\subfigure[]{
		\includegraphics[width=0.1\textwidth]{plots/bkg/feynmans/SMWW2.pdf}
		\caption{}
		\label{fig:bkg:fy_WWSM}
	}
	\subfigure[]{
		\includegraphics[width=0.1\textwidth]{plots/bkg/feynmans/SMWZ2.pdf}
		\caption{}
		\label{fig:bkg:fy_WZSM}
	}
	\subfigure[]{
		\includegraphics[width=0.1\textwidth]{plots/bkg/feynmans/stoptW.pdf}
		\caption{}
		\label{fig:bkg:fy_stoptw}
	}
	\subfigure[]{
		\includegraphics[width=0.1\textwidth]{plots/bkg/feynmans/stops.pdf}
		\caption{}
		\label{fig:bkg:fy_stops}
	}
	\subfigure[]{
		\centering
		\includegraphics[width=0.1\textwidth]{plots/bkg/feynmans/stopt.pdf}
		\caption{}
		\label{fig:bkg:fy_stopt}
	}
	\caption[Feynman diagrams of the minor background processes]{Feynman diagrams of the minor background processes. Additional background contributions that are taken into account are WW-/WZ-production with (a/b) and without (c/d) triple gauge coupling as well as single top production in association with a W boson (e), in the $s$-channel (f) and in the $t$-channel (g).}
\end{figure}