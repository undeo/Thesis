\chapter*{Introduction}
Being developed between 1960 and 1970, the Standard Model of particle physics is still the most succesful theory describing particles and their interactions. It combines the three fundamental forces given by the electromagnetic, the weak and the strong force into one consistent theory. It has been confirmed countless of times, most recently with the detection of the infamous Higgs boson, and its parameters have been measured with ever increasing accuracy.\\

Albeit its succes and reliability, there are physical aspects that are not covered by the Standard Model, like the forth fundamental force of gravity, which appears to be much weaker than the three other forces (also known as the hierachy problem). Additionally, experimental evidence in cosmology points to the existence of a new form of matter, the so-called dark matter, that only interacts via gravity and possibly the weak force. Any physics that is not described by the Standard Model is called physics beyond the Standard Model (BSM). The search for BSM physics [became the motivation] for particle physics experiments.\\

The largest and most powerful particle collider up to date is the Large Hadron Collider (LHC) in Geneva. It started operating in 2010 at a center of mass energy of 7\,TeV, collecting enough data for the discovery of the Higgs boson. The new tasks for the LHC are now given by the precise measurement of the properties of the Higgs boson and the search for BSM physics. Therefore, the center of mass energy was further increased to 13\,TeV, allowing to analyse rare processes that were inaccessible before.\\

A sector of particular interest for searches for BSM physics is the production of electroweak boson pairs with high transverse momentum ($p_{\rm T}$). The dominant channel of electroweak boson pair production includes the coupling of three gauge bosons, the so-called triple gauge coupling. The effects of possible BSM physics can be described by altering these couplings, leading to anomalous triple gauge couplings (aTGC). Introducing a new scale $\Lambda$ at which the new physics occurs, the aTGC can be parametrized in a model-independent way using the effective field theory (EFT) approach, leading to the three aTGC parameters \Tcwww \ , \Tccw \ and \Tcb . The EFT approach is only valid at energies lower than the scale $\Lambda$. It allows to search for new physics at energies that are currently not accessible to direct searches.\\

The two boson can either decay into a pair of leptons or a pair of quarks. Depending on those decays, different channels are defined. The analysis described in this thesis focusses on the semileptonic channel with one leptonically and one hadronically decaying boson. This channel offers a higher branching fraction compared to the all leptonic channel but smaller background contributions than the all hadronic channel. The event signature of the semileptonic channel [contains] one lepton from the leptonically decaying W boson and two particle showers from the hadronically decaying W or Z boson. Since the sensitivity to anomalous couplings is expected to be highest at high $p_{\rm T}$, only events with high $p_{\rm T}$ are considered. This leads to an overlapping of the two particle showers, making it neccesary to apply substructure methods to reduce background contributions. \\

The goal of this thesis is to set limits on the aTGC parameters. In contrast to previous analyses the limits are extracted using parametric models for the background and signal contributions. Using data-driven methods for the background estimation and a generic signal model, different scenarios are considered, leading to one- and two-dimensional limts. The first chapter of aims to provide the theoretical background needed for the analysis. In the second chapter the setup of the Large Hadron Collider and the Compact Muon Solenoid detector, which provided the data used in the analysis, are outlined. The reconstruction and simulation of colision events as well as the event selection are described in the third chapter. The last three chapters focus on the analysis, starting with a desription of the signal model in chapter 4, followed by a detailed illustration of the background extraction in chapter 5 and concluding with a presentation of the extracted limits on the anpmalous coupling paramters.





