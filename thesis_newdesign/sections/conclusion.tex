\chapter{Conclusion}
\label{ch:Conclusion}
Since the discovery of the Higgs boson, the last missing particle of the Standard Model, searches for physics beyond the Standard Model gained ever increasing importance. There are many theories predicting new particles at high energies that are currently unaccessible for direct searches. However, the effects of new physics might be visible at lower energies. A sector that is sensitive to BSM physics is the production of electroweak boson pairs via triple gauge coupling. Deviations from the Standard Model expectation are described in a model-independent way using the effective field theory approach, parametrizing the deviations using three anomalous coupling parameters given by \Tcwww \ , \Tccw \ and \Tcb . In this approach the assumption is made that new physics exists at a scale $\Lambda$ that is much higher than the center of mass energy of the considered events.\\

The aim of this thesis is to search for anomalous couplings that describe the effects of possible new physics on the diboson production. Therefore, data recorded at a center of mass energy of $\sqrt{s}=13$\,TeV with the CMS detector in 2015 corresponding to 2.3\,fb$^{-1}$ is used to analyse semileptonic WW and WZ boson decays at high transverse momenta. The results presented in this thesis are the first results on anomalous triple gauge couplings using data taken at 13\,TeV that were published by the CMS collaboration.\\

The events studied in this thesis contain one leptonically decaying W boson and one hadronically decaying W or Z boson. The main observable is given by the invariant mass of the diboson system, since the aTGC contributions are expected to be most significant at high invariant masses. The mass of the jet reconstructed as hadronically decaying boson is used to define a sideband and a signal region. The data in the sideband region is used for the background estimation while the data in the signal region is used in the final limit extraction. The signal region is further divided into a high- and a low-mass region, referred to as WW- and WZ-category, to gain some separation power between WW and WZ events. Since one of the aTGC parameters almost exclusively contributes to WW events, this allows to differentiate various aTGC scenarios. To take different reconstruction efficiencies and systematic uncertainties into account, the flavour of the lepton in the final state is used to define two channels, given by the electron and muon channel.\\

The main background contribution was estimated from data in a sideband region using a technique developed for resonance searches in the diboson production, the so-called alpha-ratio-method. A transfer function was extracted from simulated events and used to obtain a prediction of the background in the signal region from the observed events in the sideband region. The signal process, given by WW or WZ production, was described by a parametric function explicitly depending on the aTGC parameters, allowing to study different scenarios where either one or two aTGC parameters are unequal to zero, leading to one- and two-dimensional limits. The resulting background estimation as well as the signal model for the muon channel in the WW-category are shown in Fig.~\ref{fig:conclusion:a}.\\

To extract limits on the aTGC parameters, an unbinned maximum likelihood fit was performed to compare the background and signal estimation with the observed data, combining the WW- and WZ-category in both the electron and muon channel. A scan over a range of one or a grid of two aTGC parameters was performed and limits were extracted at 95\% confidence level (C.L.) from the likelihood distributions. Systematic uncertainties were taken into account as nuisance parameters in the likelihood function.\\

In Fig.~\ref{fig:conclusion:b} the extracted two-dimensional limits for two different parameter combinations are shown, while the extracted one-dimensional limits are summarized in Tab.\ref{tab:conclusion:1d}. As can be seen there, the observed limits are close to the expected ones. The limits found to be compatible with the SM, so no evidence for anomalous couplings has been found. The limits set in this analysis are comparable to previous results, despite the fact that only a comparatively small amount of data was used. This shows the potential of improvement for performing the analysis on the data taken by the CMS detector in 2016, corresponding to roughly ten times the amount of data used in this analysis.

\begin{table}
	\centering
	\caption[Expected and observed limits on single anomalous couplings at 95\%C.L.]{Expected and observed limits on single anomalous couplings at 95\%C.L. (other couplings set to zero).}
	\label{tab:conclusion:1d}
	\begin{tabular}{ccc}
	\hline
	aTGC              &     expected limit & observed limit\\
	\hline
	$\frac{c_{WWW}}{\Lambda ^2}$~(TeV$^{-2}$) &  [-8.73 , 8.70] &  [-9.46 , 9.42] \\
	$\frac{c_{W}}{\Lambda ^2}$~(TeV$^{-2}$)   &  [-11.7 , 11.1] &  [-12.6 , 12.0] \\
	$\frac{c_{B}}{\Lambda ^2}$~(TeV$^{-2}$)   & [-54.9 , 53.3] &  [-56.1 , 55.4] \\
	\hline
	\end{tabular}
\end{table}

\begin{figure}
    \centering
    \resizebox{\columnwidth}{!}
    {%
    \subfigure[]{
        \includegraphics[height=0.3\textheight]{../plots/bkg/mlvj/WW_mu_cwww.pdf}
        \label{fig:conclusion:a}
        }%
    \subfigure[]{
        \includegraphics[height=0.3\textheight]{../plots/results/cwww_ccw_2dlimit_deltaNLL.pdf}
        \label{fig:conclusion:b}
        }
    \caption[Comparison of data and background prediction as well as two-dimensional limits for \Tccw -\Tcb]{Comparison of data and background prediction in the invariant mass spectrum in the muon channel, WW-category (a) with the expected signal contribution for $\Tcwww=12$\,TeV$^{-2}$ (magenta, dashed), as well as two-dimensional limits for \Tccw -\Tcb \ (b). The limit plot shows the expected limits as dashed lines for 68\% C.L. (blue), 95\% C.L. (green) and 99\% C.L. and the observed limit at 95\% C.L. as black line. The observed limits are close to the expectations.}
    \label{fig:conclusion}
    }
\end{figure}