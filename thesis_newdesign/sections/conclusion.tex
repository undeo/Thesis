\chapter{Conclusion}
\label{ch:Conclusion}
Since the discovery of the Higgs boson, the last missing particle of the Standard Model, searches for physics beyond the Standard Model gained ever increasing importance. There are many theories predicting new particles at high energies unaccessible for direct searches. However, the effects of new physics might be visible at lower energies. The increased center of mass energy allows for analysing 

-search for BSM, new important sector, might show in EWK,

The aim of this thesis is to set limits on the anomalous coupling parameters that describe the effects of possible new physics on the diboson production. Therefore, 

Anomlaous couplings were parametrized in a model-independent way using the effective field theory approach. In this approach the assumption is made that new physics exists at a scale $\Lambda$ that is much higher than the center of mass energy of the considered events.

The main background contribution was estimated from data in a sideband region using a technique developed for resonance searches in the diboson production, the so-called alpha-ratio-method. The signal process, given by WW or WZ production, was [described in a parametric way] using a generic model.
To take different reconstruction efficiencies and [??] into account, the flavour of the lepton in the final state is used to define two channels, given by the electron and muon channel.
-WW/WZ category
-EFT,aTGC
-goal: limits on aTGC params
-2.3fb, 13TeV, first CMS results on aTGC

Different scenarios were studied where either one or two aTGC parameters are unequal to zero, leading to one- and two-dimensional limtis. 
-1 and 2 dim limits, 95\%CL, fig
-compare to prev results

The limits set in this analysis are comparable to previous results, despite the fact that only a comparatively small amount of data was used. In Fig.~\ref{} the extracted two-dimensional limits for two different parameter combinations are shown. As can be seen there, the observed limits are close to the expected ones, so no evidence for anomalous couplings has been found.

\begin{figure}
    \centering
    \subfigure[]{
        \includegraphics[width=0.5\textwidth]{../plots/bkg/mlvj/WW_mu_cwww.pdf}
        }%
    \subfigure[]{
        \includegraphics[width=0.4\textwidth]{../plots/results/ccw_cb_2dlimit_deltaNLL.pdf}
        }
    \caption[]{\dots}
    \label{}
\end{figure}