\chapter{Limits on Anomalous Triple Gauge Couplings}
\label{chap:LimitsonATGCs}
The main focus of this thesis lays on setting limits on the aTGC parameters in a model independent way using the effective field theory approach. Hereby the assumption was made that any new physics that may lead to anomalous couplings only exists at a scale $\Lambda$ that is much larger than the center of mass energy of the analysed collisions. By only considering events with $\MWV<3.5$\,TeV, a lower reference point for $\Lambda$ is obtained. The analysis corresponding to this thesis \cite{PAS} is the first analysis on aTGCs by the CMS collaboration that uses data taken at a center of mass energy of 13\,TeV. The used data were recorded in 2015 by the CMS detector and correspond to an integrated luminosity of 2.3\,fb$^{-1}$.\\

\noindent In this chapter, the results of the analysis are presented. Using the signal model described in chapter~\ref{chap:signal} and the background estimation from chapter~\ref{chap:bkg}, limits on the three aTGC parameters are estimated. Different scenarios where either one or two aTGC parameters are unequal to zero are studied. Additionally, the limits are computed in the vertex parametrization that was used in previous analyses (e.g. \cite{aTGC1,aTGC2}) to be able to set them in comparison with recent results.

\section{Limit Setting Procedure}
The limits on the aTGC parameters are ultimately estimated by performing an unbinned maximum likelihood fit to compare the background and signal model with the observed data, taking into account uncertainties as nuisance parameters. The likelihood is estimated as function depending on the aTGC parameters. The resulting distributions are used to extract limits on the aTGC parameters at 95\% Confidence Level (C.L.). In the first part of this section, the concept of the maximum likelihood estimation as method to compare an analytical model with measured data as well as the extraction of confidence intervals from a likelihood distribution are presented, while in the second part the final procedure of setting the limits is outlined.
\subsection{Maximum Likelihood Estimation}
\label{subsec:maxlikeest}
Given a data sample $x_1,\dots,x_n$ of a observable $x$ that is distributed according to a probability density function (pdf) $f(x_i,\theta)$ with a set of parameters $\theta=(\theta_1,\dots,\theta_m)$, the joint pdf for measuring the $x_i$ is given by \cite{cowan}
\begin{equation}
p(x_i,\theta) = \prod_{i=1}^n f(x_i,\theta) ~.
\end{equation}
The so called likelihood function is defined as the joint pdf with the $x_i$ fixed
\begin{equation}
L(\theta) = \prod_{i=1}^n f(x_i,\theta) ~.
\end{equation}
It is expected to yield high values for parameters values $\theta$ close to the true ones. The maximum likelihood estimators $\hat{\theta}$ are given by the parameter values that maximize the likelihood function
\begin{equation}
\left. \frac{\partial L}{\partial \theta_i} \right| _{\theta_i=\hat{\theta_i}} = 0 ~.
\label{eq:limits:1stderivlikelihood}
\end{equation}

Instead of maximizing $L$ it is convenient to minimize $-L$. It is also conventional to use the logarithm of the likelihood function, which allows to add different likelihoods instead of multiplying them. In the following, $L$ refers to the logarithmic likelihood.

\subsection{Confidence Intervals}
A confidence interval is an alternative way of reporting statistical errors on a measurement. If the measurement of a parameter $c$ was repeated many times, a confidence interval $[a,b]$ would include the true value of $c$ in a fraction $1-\alpha-\beta$ of the experiments, where $\alpha$ and $\beta$ are the probabilities of measuring a value lower than $a$ or higher than $b$, respectively.\\

Confidence intervals can be directly extracted from the logarithmic likelihood function $L(\theta)$. Therefore, it is expanded it in a Taylor series about the maximum likelihood estimate $\hat{\theta}$ \cite{cowan}
\begin{equation}
L(\theta) = L(\hat{\theta}) + \left. \frac{\partial L}{\partial \theta}\right|_{\theta=\hat{\theta}}(\theta-\hat{\theta}) + \frac{1}{2}\left.\frac{\partial^2L}{\partial \theta^2}\right|_{\theta=\hat{\theta}}(\theta-\hat{\theta})^2 + \mathcal{O}\left((\theta-\hat{\theta})^3\right) ~.
\label{eq:limits:liketaylor}
\end{equation}
The first derivative of the likelihood vanishes at $\hat{\theta}$ (equation~\ref{eq:limits:1stderivlikelihood}). Using the Cram\'{e}r-Rao-bound and neglecting terms of higher order leads to \cite{cowan}
\begin{equation}
L(\hat{\theta}\pm N\sigma) = L_{\rm max} - \frac{N^2}{2} 
\end{equation}
in case of a Gaussian likelihood function. For non-Gaussian likelihood functions the confidence interval gets asymmetric, leading to
\begin{equation}
L(\hat{\theta}_{-c}^{+d}) = L_{\rm max} - \frac{N^2}{2}
\end{equation}
with the confidence interval $[c,d]$. The confidence level depends on the value of $N$, with $N=1.92$ corresponding to a 95\% C.L.

\subsection{Limit Setting Procedure}
\label{subsec:limsetproc}
The limits on the aTGC parameters are estimated by comparing the background and signal prediction with the observed data in the WW- and WZ-category for the electron and muon channel. For this, a limit setting tool created for the combination of the Higgs measurements of CMS and ATLAS \cite{combine} is used. It is capable of estimating and combining the likelihood functions of several channels depending on one or more parameters of interest (POI), taking into account uncertainties and their correlations as nuisance parameters. A scan over the POIs is performed, evaluating the likelihood for each POI value (or each set of POI values in case of two-dimensional limits) by maximizing it with respect to the nuisance parameters. The limits are ultimately extracted from the delta-log-likelihood distributions, which is defined as the difference of the logarithmic likelihood for a given POI value and the logarithmic likelihood for the best fit point. The best fit point is estimated by maximizing the likelihood function with respect to the POIs and the nuisance parameters simultaneously.\\

\noindent The likelihood function for a single channel and a single POI $c$ is given by
\begin{equation}
\mathcal{L} = \frac{1}{k} \left( S(c)\cdot f_{\rm aTGC}(c) + B \cdot f_{\rm bkg} \right) e^{S(c) + B} \cdot \rho(\theta)~,
\end{equation}
where $k$ denotes the total number of observed events, $S(c)$ the expected signal yield depending on the POI $c$, and $B$ the expected background event yield. $f_{\rm aTGC}(c)$ and $f_{\rm bkg}$ are given by the pdfs describing the signal and the background processes, respectively, while  $\rho(\theta)$ describes the set of uncertainties $\theta$. In this analysis, the signal yield $S(c)$ and the signal pdf $f_{\rm aTGC}(c)$ depend both on $c$ and $c^2$, according to the signal yield and shape introduced in chapter~\ref{chap:signal}.\\

\noindent There are two kinds of uncertainties implemented as nuisance parameters $\theta$. Uncertainties on the normalizations of the different contributions are implemented as multiplicative parameters constrained by log-normal distributions. Uncertainties affecting the slope parameters of the background and signal estimations are implemented by adding a Gaussian term for each parameter, centred at the nominal parameter value and having a width corresponding to the parameter uncertainty.\\

\noindent To be able to interpret the limits extracted from the observed data, they have to be compared with the expected limits that would result in the case of a vanishing signal contribution. Therefore, a pseudo-dataset is generated using only the pdfs describing the background distributions. Expected limits are then extracted as described above.


\section{Systematic Uncertainties}
\label{sec:systematics}
There are several sources of systematic uncertainties arising from the event reconstruction that affect the resulting event yields and the shapes of the background and signal distributions. In the following section the considered systematic uncertainties are shortly described. They were not evaluated in the context of this thesis but rather in the corresponding physics analysis \cite{PAS}. These uncertainties are not applied to the W+jets background contribution since its uncertainty is directly extracted from the measured data in the background extraction.

\subsection*{Uncertainties on the Normalizations}
In this section the different sources  of systematic uncertainties are outlined. The resulting uncertainties on the normalizations are included in the likelihood functions as nuisance parameters, taking into account correlations between the channels. They are summarized in Tab.~\ref{tab:limits:systs}. As can be seen there, the \ttbar \ contribution is dominated by the uncertainty resulting from the renormalization and factorization scale, while the single top and diboson contributions are dominated by the $\tau_{21}$ cut efficiency.
\subsubsection*{Luminosity}
The simulated background samples as well as the signal sample are initially normalized according to the integrated luminosity of the used data sample. The uncertainty of the luminosity measurement of $2.7$\% is included as nuisance parameter correlated between all channels.
\subsubsection*{b quark reconstruction}
To account for the efficiency in identifying jets coming from b quark decays \cite{CSV2} an uncertainty is applied on the \ttbar \ and the WZ contribution. The uncertainty resulting from misidentifying jets as originating from b quarks is found to be negligible.
\subsubsection*{Jet energy scale}
The effects of varying the jet energy scale on the normalizations of the different processes is propagated as additional uncertainty, correlated between all channels.
\subsubsection*{N-subjettiness cut efficiency}
To account for the efficiency of the cut on the N-subjettiness $\tau_{21}$ a correlated uncertainty of 12\% on all processes containing hadronic boson decays is applied. For the single top production this uncertainty is not applied to the $t$- and $s$-channel processes, since here the hadronic boson candidate is given by a jet originating from a gluon or light quark.
\subsubsection*{Lepton energy scale, energy resolution and efficiency}
The effects of the lepton energy scale, the resolution of the energy reconstruction as well as the reconstruction efficiency are included as uncertainty on all processes.
\subsubsection*{Missing transverse energy}
Unclustered energy deposits in the detector that are not assigned to any particle can be falsely assigned to the \MET . From this an uncertainty is estimated and propagated to the normalizations. 
\subsubsection*{Parton distribution functions (PDF)}
The PDFs that describe the energy distribution between the partons of the protons are empirically determined. The uncertainty of the corresponding measurement is evaluated and added as uncertainty.
\subsubsection*{Renormalization and factorization scale}
The theoretical cross section of the different considered processes are calculated only up to a certain order of perturbation (leading order or next-to-leading order). Neglecting the higher order terms introduces divergencies that can be avoided by introducing a new scale $Q^2$. The effects of increasing or decreasing $Q^2$ by a factor of 2 are estimated for the different background contributions and added as uncertainty. 

\begin{table}
    \centering
    \caption[Summary of the estimated systematic uncertainties on the normalizations]{Summary of the estimated systematic uncertainties on the normalizations in \%, values taken from \cite{PAS}. Shown are the applied uncertainties corresponding to jet energy scale (JES), b quark reconstruction efficiency (b-tag), N-subjettiness cut efficiency ($\tau_{21}$ eff.), lepton energy scale (lep. en.), lepton efficiency (lep. eff.), missing transverse energy (\MET), luminosity (lumi), parton distribution functions (PDF) and renormalization and factorization scale ($Q^2$). The total uncertainty is estimated by adding the different contributions in quadrature.}
    \label{tab:limits:systs}
    \resizebox{0.95\columnwidth}{!}
    {%
    \begin{tabular}{lrrrrrrrrrr}
    \hline
    process     & JES      & b-tag     & $\tau_{21}$ eff.      & lep. en.      & lep. eff.      & \MET      & lumi      & PDF       & $Q^2$     & total \\
    \hline
    \multicolumn{11}{c}{electron channel} \\
    \ttbar      & 2.8           & 0.8      & 12              & <0.5         & 1.0          & 0.5        & 2.7       & 2.5         & 19       & 23\\
    diboson     & 2.4           & <0.5     & 12              & 0.6          & 1.0          & 0.6        & 2.7       & 2.5         & 6.0       & 14\\
    single top  & 1.6           & <0.5     & 12              & 0.5          & 1.0          & 1.2        & 2.7       & 0.3         & 2.0       & 13\\
    \hline
    \multicolumn{11}{c}{muon channel} \\
    \ttbar      & 2.6           & 0.8      & 12              & 1.6          & 3.2          & <0.5       & 2.7       & 2.6         & 19       & 23\\
    diboson     & 2.3           & <0.5     & 12              & 1.7          & 3.9          & <0.5       & 2.7       & 2.3         & 6.0       & 15\\
    single top  & 0.6           & <0.5     & 12              & 1.9          & 3.6          & 0.5        & 2.7       & <0.5        & 1.9       & 13\\
    \hline
    \end{tabular}
    }
\end{table}

\subsection*{Uncertainties on the Slope Parameters}
\label{sec:uncslopesig}

The uncertainties mentioned above do also affect the shape of the signal distribution. These effects are propagated to the limit estimation as uncertainties on the signal slope parameters. They were also not estimated as part of this thesis but taken from \cite{PAS}.\\

\noindent To estimate the effect of the systematic uncertainties on the values of the slope parameters of the signal function, the parameter extraction is redone as described in chapter~\ref{chap:signal}. For the sake of simplicity, the interference terms in equation~\ref{eq:signal:sigfunc} are neglected, leading to only three slope parameters $a_{c_{WWW}}$, $a_{c_W}$ and $a_{c_B}$. For each systematic uncertainty, a new signal sample is created by varying the source of the uncertainty. The uncertainties on the slope parameters are evaluated as difference between the nominal parameter values and the values extracted from the varied samples. The resulting uncertainties are dominated by the contributions from the PDF as well as renormalization and factorization scale, while in the WZ-category the uncertainty on $a_{c_B}$ is dominated by the contribution corresponding to the jet energy scale (JES). This is caused by the fact that \Tcb \ mostly contributes to the WW-category. A variation in the JES leads to migration between the two categories, thus affecting the resulting shape.\\

\noindent The final uncertainty on the slope parameters is estimated by adding the different contributions in quadrature, leading to an uncertainty of 5\% that is applied to all signal shape parameters except for the one corresponding to \Tcb \ , for which an uncertainty of 15\% is applied.\\

\noindent The uncertainty on the parameters describing the shape of the main background contribution, given by W+jets production, is taken from the fit result of the background extraction in section~\ref{sec:shapewjets}. To account for possible different parametrizations of the W+jets distribution (e.g. the alternative function given in equation~\ref{eq:bkg:altfunc}), the uncertainties are increased by a factor of $\sqrt{2}$. For the second largest background contribution, given by \ttbar \ production, the parametric fit uncertainties are also included in the limit extraction. Since the effects of the systematic uncertainties mentioned above are not explicitly evaluated, the fit uncertainty is artificially increased by a factor of 2.


\section{One-Dimensional Limits}
\begin{table}
	\centering
	\caption[Comparison of the background and signal yields with the observed data in the WW- and WZ-categories]{Comparison of the background and signal yields with the observed data in the WW- and WZ-categories for both the electron and muon channel. Uncertainties for the single-top, diboson and \ttbar \ contributions are evaluated as described in section~\ref{sec:systematics}, while the uncertainty on W+jets is obtained in the background extraction in section~\ref{sec:bkgnorms}.}
	\label{tab:limits:bkgnorms}
	\begin{tabular}{cr@{\,}c@{\,}lr@{\,}c@{\,}lr@{\,}c@{\,}lr@{\,}c@{\,}l}
		\hline
		& \multicolumn{6}{c}{electron channel} & \multicolumn{6}{c}{muon channel} \\
		category  & \multicolumn{3}{c}{WW} & \multicolumn{3}{c}{WZ} &\multicolumn{3}{c}{WW} & \multicolumn{3}{c}{WZ} \\
		\Mpr ~(GeV)& \multicolumn{3}{c}{[65,85]} & \multicolumn{3}{c}{[85,105]} &\multicolumn{3}{c}{[65,85]} & \multicolumn{3}{c}{[85,105]} \\
		\hline
		W+jets     		   & 124  &$\pm$& 17   & 103 &$\pm$& 16   & 192  &$\pm$& 20   & 164  &$\pm$& 20  \\ 
		$\ttbar$            & 73   &$\pm$& 17   & 58  &$\pm$& 13   & 90   &$\pm$& 21   & 71   &$\pm$& 17 \\
		single top 		   & 10.9 &$\pm$& 1.4  & 9.8 &$\pm$& 1.2  & 17.8 &$\pm$& 2.3  & 10.6 &$\pm$& 1.4  \\
        diboson  (SM) 	   & 15.8 &$\pm$& 2.2  & 9.3 &$\pm$& 1.3  & 20.6 &$\pm$& 3.0  & 12.2 &$\pm$& 1.8 \\
        Total expected (SM) & 224  &$\pm$& 24   & 180 &$\pm$& 21   & 320  &$\pm$& 29   & 258  &$\pm$& 26 \\
        \hline
        diboson $\frac{c_{WWW}}{\Lambda^2}=12\, {\rm TeV}^{-2}$ & 36.2 & $\pm$& 5.1 & 39.9 &$\pm$& 5.7 & 50.8 &$\pm$& 7.3 & 55.4 &$\pm$& 8.0 \\
		diboson $\frac{c_{W}}{\Lambda^2}=20\, {\rm TeV}^{-2}$   & 52   & $\pm$& 7   & 69   &$\pm$& 10  & 72   &$\pm$& 10  & 91   &$\pm$& 13 \\
		diboson $\frac{c_{B}}{\Lambda^2}=60\, {\rm TeV}^{-2}$   & 41.5 & $\pm$& 5.9 & 20.1 &$\pm$& 2.9 & 57.0 &$\pm$& 8.2 & 26.8 &$\pm$& 3.9 \\
		\hline
        Data   & 234 & & & 183 & & & 340 & & & 265 & &\\
        \hline
	\end{tabular}
\end{table}

One-dimensional limits are computed by performing a scan over one of the aTGC parameters while keeping the remaining two set to zero. For each scanned point the delta-log-likelihood ($\Delta NNL$) is computed as described in section~\ref{subsec:limsetproc}, combining the WW- and WZ-category in both the electron and muon channel. The final background and signal normalizations in each channel together with the applied uncertainties are summarized in Tab.~\ref{tab:limits:bkgnorms}.\\

\noindent The limits at 95\% C.L. on the three aTGC parameters are extracted from the $\Delta NLL$ contribution at $\Delta NLL=1.92$. For the expected limits a pseudo-dataset was generated under the assumption af a vanishing signal contribution, i.e. all aTGC parameters were set to zero. The resulting expected and observed limits are summarized in Tab.~\ref{tab:limits:1dlimits}. The $\Delta NLL$ distributions are shown in Fig.~\ref{fig:limits:1dlimits}. The limits are found to be slightly asymmetric, which is caused by sign-sensitive terms in the signal function in equation~\ref{eq:signal:sigfunc}, representing the interference of the aTGC with the SM process. The observed limits at 95\% C.L. are not as strict as the expected ones. This is can be explained by an overfluctuation in the data, i.e. more events than expected are observed. However, the sensitivity of this analysis is not driven by the total event yield, which is not significantly affected by the anomalous couplings. It rather comes from events at high invariant masses where the difference between signal and background is expected to be highest. The fact that the limits for \Tcb \ are closer to the expectation than the ones for \Tcwww \ and \Tccw \ indicates that there is an overfluctuation of high-mass events in the WZ-category, where the sensitivity to \Tcb \ is comparatively low. However, the observed limits are still within the $\pm 1\sigma$-environment of the expected limits, which can be approximated by the limits at 68\% and 99\% C.L., thus no evidence of anomalous couplings has been found. A comparison of the limits with the results of previous analyses can be found in section~\ref{sec:vertex}.\\

\begin{table}
	\centering
	\caption[Expected and observed limits on single anomalous couplings]{Expected and observed limits on single anomalous couplings (other couplings set to zero).}
	\label{tab:limits:1dlimits}
	\begin{tabular}{ccccc}
	\hline
	\multirow{2}{*}{aTGC}              &     \multicolumn{3}{c}{expected limit} & observed limit\\
	& 68\% C.L. & 95\% C.L. & 99\% C.L. & 95\% C.L.\\
	\hline
	$\frac{c_{WWW}}{\Lambda ^2}$~(TeV$^{-2}$) & [-6.51 , 6.49] & [-8.73 , 8.70] & [-9.97 , 9.94] & [-9.46 , 9.42] \\
	$\frac{c_{W}}{\Lambda ^2}$~(TeV$^{-2}$)   & [-8.8 , 8.1]   & [-11.7 , 11.1] & [-13.3 , 12.8] & [-12.6 , 12.0] \\
	$\frac{c_{B}}{\Lambda ^2}$~(TeV$^{-2}$)   & [-40.5 , 38.7] & [-54.9 , 53.3] & [-63.2 , 61.7] & [-56.1 , 55.4] \\
	\hline
	\end{tabular}
\end{table}

\begin{figure}
	\centering
	\subfigure[]{
		\includegraphics[width=0.59\textwidth]{../plots/results/1dlimit_cwww.pdf}
	}
	\subfigure[]{
		\includegraphics[width=0.59\textwidth]{../plots/results/1dlimit_ccw.pdf}
	}
	\subfigure[]{
		\includegraphics[width=0.59\textwidth]{../plots/results/1dlimit_cb.pdf}
	}		
	\caption[Delta-log-likelihood distributions for the three aTGC parameters]{$\Delta NNL$ distributions for \Tcwww \ (a), \Tccw \ (b) and \Tcb \ (c). Shown are the expected (green straight line) and observed distributions (black straight line). The intersection of the black horizontal dashed line and the expected or observed distribution corresponds to the 95\% C.L., indicated by the vertical green and black dashed lines for the expected and observed limit, respectively. Also shown is the $\pm 1\sigma$-environment around the expected limit (shaded area). The observed limits agree with the expected ones within the $\pm 1\sigma$-environment.}
	\label{fig:limits:1dlimits}
\end{figure}

\section{Two-Dimensional Limits}
\label{sec:2dlims}
The extraction of two-dimensional limits is performed similarly to the one-dimensional ones, except only one aTGC parameter is set to zero. This leads to a two-dimensional $\Delta NLL$ distribution. Limits at 95\% C.L. are extracted as two-dimensional contours in the $\Delta NLL$ distribution as shown in Fig.~\ref{fig:limits:2dlimits}.\\

\noindent Similarly to the one-dimensional case, the observed two-dimensional limits are not as strict as the expected ones but still within the $\pm 1\sigma$-environment, which again can be approximated by the contours at 68\% and 99\% C.L.\\

\noindent The $\Delta NLL$-contour for limits involving \Tcb \ are found to be non-elliptical. This is caused by the different sensitivities of the aTGC parameters to the WW- and WZ-category and is described in more detail in section~\ref{sec:limits:atgcint}. Another effect on the shapes of two-dimensional limits are the interfenrence effects between the different aTGC contributions that were introduced in section~\ref{subsec:signal:aTGCInt}, which causes the contours to be slightly rotated counterclockwise. These effects are also described in more detail in section~\ref{sec:limits:atgcint}.

\begin{figure}
	\centering
	\subfigure[]{
		\includegraphics[width=0.5\textwidth]{../plots/results/cwww_ccw_2dlimit_deltaNLL.pdf}
	}%
	\subfigure[]{
		\includegraphics[width=0.5\textwidth]{../plots/results/cwww_cb_2dlimit_deltaNLL.pdf}
	}
	\subfigure[]{
		\includegraphics[width=0.5\textwidth]{../plots/results/ccw_cb_2dlimit_deltaNLL.pdf}
	}		
	\caption[Two-dimensional delta-log-likelihood distributions for the three combinations of aTGC parameters]{Two-dimensional delta-log-likelihood distributions for \Tcwww -\Tccw \ (a), \Tcwww -\Tcb \ (b) and \Tccw -\Tcb\ (c). Shown are the expected contours (dashed lines) corresponding to the 68\% (blue), 95\% (green) and 99\% C.L. (red) as well as the contour corresponding to the observed 95\% C.L. (black line).}
	\label{fig:limits:2dlimits}	
\end{figure}

\section{Limits in Vertex Parametrization}
\label{sec:vertex}
Describing the effects of anomalous couplings with aTGC parameters using the effective field theory approach is a novel way of analysing triple gauge couplings. Previous analyses used the vertex parametrization commonly used at LEP \cite{LEPatgc}, where the aTGC are described by the couplings in the SM Lagrangian given in equation~\ref{eq:theo:EWKlag}. The limits resulting from these analyses cannot be simply converted to the EFT parametrization. However, the generic signal model introduced in chapter~\ref{chap:signal} can be transformed via
\begin{align}
\cwww &= \frac{2}{3g^2M_W^2}\lambda_Z ~, \\
\ccw  &= \frac{2}{M_Z^2}\Delta g_1^Z ~, \\
\cb   &= \frac{2}{\tan^2\theta_W M_Z^2}\Delta g_1^Z - \frac{2}{\sin^2\theta_W M_Z^2}\Delta\kappa_Z ~,
\end{align}
with the W boson mass $M_W$, the Z boson mass $M_Z$, the weak coupling constant $g$ and the Weinberg angle $\theta_W$. This allows for the extraction of the limits in terms of $\lambda_Z$, $\Delta \kappa_Z$ and $\Delta g_1^Z$. The results for the one-dimensional limits are summarized in Tab.~\ref{tab:limits:1dlimits_vertex}, while the two-dimensional limits are shown in Fig.~\ref{fig:limits:2dlimitsvertex}. A comparison with the previous limits using data taken at a center of mass energy of 7\,TeV is shown in Fig.~\ref{fig:limits:compare}. As can be seen there, the limits extracted in this analysis are competitive to the previous ones, despite using a comparatively small dataset of 2.3\,fb$^{-1}$. Running this analysis on the dataset coming available at the end of 2016, which corresponds to $38.27$\,fb$^{-1}$, is expected to significantly improve the limits.

\begin{table}
	\centering
	\caption[Expected and observed limits on single anomalous couplings in the vertex parametrization]{Expected and observed limits on single anomalous couplings in the vertex parametrization (other couplings set to zero).}
	\label{tab:limits:1dlimits_vertex}
	\begin{tabular}{ccccc}
	\hline
	\multirow{2}{*}{aTGC}              &     \multicolumn{3}{c}{expected limit} & observed limit\\
	& 68\% C.L. & 95\% C.L. & 99\% C.L. & 95\% C.L.\\
	\hline
	$\lambda$          & [-0.027 , 0.027] & [-0.036 , 0.036] & [-0.041 , 0.041] & [-0.039 , 0.039] \\
	$\Delta g_1^Z$     & [-0.048 , 0.046] & [-0.066 , 0.064] & [-0.075 , 0.074] & [-0.067 , 0.066] \\
	$\Delta\kappa_Z$   & [-0.028 , 0.029] & [-0.038 , 0.040] & [-0.045 , 0.046] & [-0.040 , 0.041] \\
	\hline
	\end{tabular}
\end{table}

\begin{figure}
	\centering
	\subfigure[]{
		\includegraphics[width=0.5\textwidth]{../plots/results/lZ_dkz_2dlimit_deltaNLL.pdf}
	}%
	\subfigure[]{
		\includegraphics[width=0.5\textwidth]{../plots/results/lZ_dg1z_2dlimit_deltaNLL.pdf}
	}
	\subfigure[]{
		\includegraphics[width=0.5\textwidth]{../plots/results/dkz_dg1z_2dlimit_deltaNLL.pdf}
	}		
	\caption[Two-dimensional delta-log-likelihood distributions for the three combinations of aTGC parameters in the vertex parametrization]{Two-dimensional delta-log-likelihood distributions for $\lambda_Z$-$\Delta\kappa_Z$ (a), $\lambda_Z$-$\Delta g_1^Z$ (b) and $\Delta\kappa_Z$-$\Delta g_1^Z$ (c). Shown are the expected distributions (dashed lines) for the 68\% (blue), 95\% (green) and 99\% C.L. (red) as well as the observed distribution (black line).}
	\label{fig:limits:2dlimitsvertex}	
	\label{fig:limits:compare}
\end{figure}

\begin{figure}
    \centering
    \includegraphics[width=0.9\textwidth]{../plots/results/compare.pdf}
    \caption[Comparison of the limits on the aTGC parameters in vertex parametrization with previous results]{Comparison of the limits on the aTGC parameters in vertex parametrization with previous results, adapted from \cite{limits_compare}. Shown are the results of analyses by the CMS (red) and ATLAS collaboration (blue) as well as combined limits by LEP (black) and D0 (green). The limits taken in this analysis are highlighted in yellow. As can be seen here, the limits are comparable with the previous ones, despite the comparatively small used dataset.}
\end{figure}

\section{Additional Verification of the Limit Setting Procedure}
\label{sec:limits:atgcint}
\subsubsection*{Two-dimensional Limits without aTGC-Interference}
\begin{figure}
	\centering
	\subfigure[]{
		\includegraphics[width=0.5\textwidth]{../plots/results/ccw_cb_2dlimit_deltaNLL.pdf}
	}%
	\subfigure[]{
		\includegraphics[width=0.5\textwidth]{../plots/results/ccw_cb_2dlimit_deltaNLL_noatgcint.pdf}
	}
	\caption[Comparison of the two-dimensional limits for \Tccw -\Tcb \ with and without the aTGC interference]{Comparison of the two-dimensional limits for \Tccw -\Tcb \ with (a) and without (b) the aTGC interference. The interference causes the contours to slightly shift clockwise.}
	\label{fig:limits:noatgcint}
\end{figure}
The effects of the interfenrence between different aTGC contributions can be illustrated in terms of a simple counting experiment with two aTGC parameters $c_1$ and $c_2$, neglecting the SM contribution. The number of observed events $N_{\rm obs}$ is then given by
\begin{equation}
N_{\rm obs} = \left( c_1A_1 + c_2A_2 \right)^2 = c_1^2A_1^2 + c_2^2A_2^2 + c_1c_2A_1A_2 ~,
\end{equation}
where $A_1$ and $A_2$ denote the contributions of the two parameters $c_1$ and $c_2$, respectively. The term describing the interference between the two parameters can be eliminated by a rotation
\begin{eqnarray}
c_1'A_1' &=& c_1A_1\cos \alpha + c_2A_2\sin \alpha \\
c_2'A_2' &=& -c_1A_1\sin \alpha + c_2A_2\cos \alpha
\end{eqnarray}
around the angle $\alpha$, which can be interpreted as the strength of the interference. This causes the two-dimensional $\Delta NLL$ contours to be rotated depending on
the size of $\alpha$. The interference effects between the different aTGC contributions are found to be comparatively small (see Tab.~\ref{tab:signal:relcoef}) and only lead to a visible rotation for the \Tccw - \Tcb -limits. In Fig.~\ref{fig:limits:noatgcint} the limits for \Tccw -\Tcb \ are shown with and without the aTGC interference.


\subsubsection*{Merging the WW- and WZ-category}

In the two-dimensional limits the different sensitivities of the aTGC parameters in the two regions are expected to affect the shape of the resulting limits. This motivated the division of the signal region into a low- and high-mass region (see section~\ref{sec:wwwzregs}). Roughly equal sensitivities in both regions for both parameters (i.e. for the two-dimensional \Tcwww -\Tccw -limits) lead to elliptical limits. In the hypothetical case of two parameters that each only contribute to one of the regions the contours of the $\Delta NLL$ distribution would take the shape of a rectangle. This is partly realised in the limits including \Tcb \ , which only has a small contribution to the WZ-category, leading to a contour shape intermediate between an ellipse and a rectangle.\\

\noindent The deformation of the $\Delta NLL$-contours for the two-dimensional limits involving \Tcb \ is caused by the division of the signal region into the WW- and WZ-category. To confirm this, the limits were estimated without dividing the signal region. The resulting $\Delta NLL$-contours are elliptical, as can be seen in Fig.~\ref{fig:limits:2dlimits_nocat}.

\begin{figure}
	\centering
	\subfigure[]{
		\includegraphics[width=0.5\textwidth]{../plots/results/cwww_ccw_2dlimit_deltaNLL_WV.pdf}
	}%
	\subfigure[]{
		\includegraphics[width=0.5\textwidth]{../plots/results/cwww_cb_2dlimit_deltaNLL_WV.pdf}
	}
	\subfigure[]{
		\includegraphics[width=0.5\textwidth]{../plots/results/ccw_cb_2dlimit_deltaNLL_WV.pdf}
	}		
	\caption[Two-dimensional delta-log-likelihood distributions for the three combinations of aTGC parameters without division of the signal region]{Two-dimensional delta-log-likelihood distributions for \Tcwww -\Tccw \ (a), \Tcwww -\Tcb \ (b) and \Tccw -\Tcb\ (c) without division of the signal region. Shown are the expected distributions (dashed lines) for the 68\% (blue), 95\% (green) and 99\% C.L. (red) as well as the observed distribution (black line).}
	\label{fig:limits:2dlimits_nocat}	
\end{figure}
