\chapter{Limits on Anomalous Triple Gauge Couplings}
\label{chap:LimitsonATGCs}


In this chapter, the results of the analysis are presented. Using the signal model described in chapter~\ref{chap:signal} and the background estimation from chapter~\ref{chap:bkg}, limits on the three aTGC parameters are estimated. Different scenarios where either one or two aTGC parameters are unequal to zero are studied. Additionally, the limits are computed in the vertex parametrization, which was used in previous analyses (e.g. \cite{aTGC1,aTGC2}).


\section{Limit Setting Procedure}
The limits on the aTGC parameters are ultimately estimated by performing an unbinned maximum likelihood fit to compare the background and signal model with the observed data, taking into account uncertainties as nuisance parameters. The likelihood is estimated depending on the aTGC parameters. The resulting distributions are used to extract limits on the aTGC parameters at 95\% Confidence Level (C.L.). In the first part of this section, the concept of the maximum likelihood estimation as method to compare an analytical model with measured data as well as confidence intervalls is presented, while in the second part the final procedure of setting the limits is outlined.
\subsection{Maximum Likelihood Estimation}
[likelihood definition]
[log likelihood]
[profiling]
\subsection{Confidence Intervals}
\subsection{Limit Setting Procedure}

For this, a limit setting tool \cite{combine} created for the combination of the different channels of the Higgs measurement \cite{higgs_comb} is used. It is capable of estimating and combining the likelihood function of several channels depending on one or more paramters of interest (POI), taking into account nuisance parameters by profiling them as explained in section~\ref{subsec:maxlikeest}. The so-called best fit point is estimated by minimizing the likelihood function with respect to the POIs
[expected limits for sensitivity]

\section{Systematic Uncertainties}
\label{sec:systematics}
There are several sources of systematic uncertainties arising from the event reconstructino that effect the resulting event yield. These uncertainties are included in the likelihood functions as nuisance parameters, taking into account correlations between the channels. In the following the considered systematic uncertainties are shortly described. They were evaluated in the context of the corresponding physics analysis \cite{PAS} and are summarized in Tab.~\ref{tab:limits:systs}. These uncertainties are not applied to the W+jets background contribution since it is directly extracted from data.
\begin{itemize}
\item Lumiosity\\
The simulated background samples as well as the signal sample are initially normalized according to the integrated luminosity of the used data sample. The uncertainty of the luminosity measurement of $2.3$\% is included as nuisance parameter correlated between all channels.
\item b quark reconstruction\\
To account for the efficiency in identifying jets coming from b quark decays \cite{bquarkuncert} an uncertainty is applied on the \ttbar \ and the WZ contribution. The uncertainty resulting from misidentifying jets as originating from b quarks is found to be negligible.
\item Jet energy scale\\

\item N-subjettiness\\
\item Lepton energy scale and resolution\\
\item Missing transverse energy\\
\item Parton distribution functions\\
\item $Q^2$\\
\end{itemize}
\begin{table}
    \centering
    \caption[Summary of the estimated systematic uncertainties on the normalizations]{Summary of the estimated systematic uncertainties on the normalizations, values taken from \cite{PAS}}
    \begin{tabular}{lrrrrrrrrrr}
    \hline
    process     & jet en.       & b-tag     & $\tau_{21}$       & lep. en.      & lep. ID      & \MET      & lumi      & PDF       & scale     & total \\
    \hline
               & \multicolun{10}{c}{electron channel} \\
    \hline
    \ttbar      & 2.8           & 0.8      & 12              & <0.5         & 1.0          & 0.5        & 2.7       & 2.5         & 19       & 23\\
    diboson     & 2.4           & <0.5     & 12              & 0.6          & 1.0          & 0.6        & 2.7       & 2.5         & 6.0       & 14\\
    Single Top  & 1.6           & <0.5     & 12              & 0.5          & 1.0          & 1.2        & 2.7       & 0.3         & 2.0       & 13\\
    \hline
               & \multicolun{10}{c}{muon channel} \\
    \hline
    \ttbar      & 2.6           & 0.8      & 12              & 1.6          & 3.2          & <0.5       & 2.7       & 2.6         & 19       & 23\\
    diboson     & 2.3           & <0.5     & 12              & 1.7          & 3.9          & <0.5       & 2.7       & 2.3         & 6.0       & 15\\
    Single Top  & 0.6           & <0.5     & 12              & 1.9          & 3.6          & 0.5        & 2.7       & <0.5        & 1.9       & 13\\
    \hline
    \end{tabular}
\end{table}
\subsection*{Uncertainties on the Normalizations}
\subsection*{Uncertainties on the Slope Parameters}
\label{sec:uncslopesig}
The uncertainties on the slope parameters of the signal function were not estimated as part of this thesis but rather in the context of a soon to be published PhD thesis \cite{IVAN}.\\

\noindent The un
[uncertainties only for norms]
[also impact on slopes]
[extract as in section signal with MC $\pm$1sigma]
[results]
[high uncert for cb -> JEC]


\section{One-Dimensional Limits}
One-dimensional limits are computed by performing a scan over one of the aTGC parameters while keeping the remaining two set to zero. For each scanned point the profiled logarithmic likelihood is computed.
[profiling of uncertainties]
Then, the delta-log-likelihood is formed, which is given by the difference of the likelihood of each point and the best fit point. The best fit point is the parameter value that maximizes the likelihood. 


\section{Two-Dimensional Limits}
\label{sec:2dlims}
\dots \\
\noindent In the two-dimensional limits the different sensitivities of the aTGC parameters in the two regions are expected to affect the shape of the resulting limits. This motivated the division of the signal region into a low- and high-mass region (see section~\ref{sec:wwwzregs}. If the sensitivities are roughly equal in both regions for both parameters (i.e. for the two-dimensional \Tcwww -\Tccw -limits), 


\section{Limits in Vertex Parametrization}
\label{sec:vertex}


\section{Additional Verification of the Limit Setting Procedure}
[one signal category]
[]
