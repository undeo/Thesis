\chapter{Limits on Anomalous Triple Gauge Couplings}
\label{chap:LimitsonATGCs}
This chapter presents the results of the analysis. Using the signal model described in section~\ref{chap:signal} and the background estimation from section~\ref{chap:bkg}, limits on the three aTGC parameters introduced in section~\ref{sec:aTGC} are estimated. Different scenarios where either one or two aTGC parameters are unequal to zero are studied. Additionally, the limits are computed in the vertex parametrization, which was used in previous analyses (e.g. \cite{aTGC1,aTGC2}).
\section{Limit Setting Procedure}
The limits on the aTGC parameters are ultimately estimated by performing a unbinned maximum likelihood fit
\subsection{Maximum Likelihood Estimation}
\subsection{Confidence Intervals}
\section{Systematic Uncertainties}
\label{sec:systematics}
\subsection{Uncertainties on the Slope Parameters}
\label{sec:uncslopesig}
The uncertainties on the slope parameters of the signal function were not estimated as part of this thesis but rather in the context of a soon to be published PhD thesis \cite{IVAN}.
\section{One-Dimensional Limits}
One-dimensional limits are computed by performing a scan over one of the aTGC parameters while keeping the remaining two set to zero. For each scanned point the profiled likelihood is computed. Then, the delta-log-likelihood is formed, which is given by the difference of the likelihood of each point and the best fit point. The best fit point is the parameter value that maximizes the likelihood.  
\section{Two-Dimensional Limits}
\label{sec:2dlims}
\dots \\
\noindent In the two-dimensional limits the different sensitivities of the aTGC parameters in the two regions are expected to affect the shape of the resulting limits. If the sensitivities are roughly equal in both regions for both parameters (i.e. for the two-dimensional \Tcwww -\Tccw -limits), 
\section{Limits in Vertex Parametrization}
\label{sec:vertex}