\chapter*{Zusammenfassung}
Nach seiner Entwicklung zwischen 1960 und 1970 ist das Standardmodell der Teilchenphysik bis heute die erfolgreichste und zuverlässigste Theorie zur Beschreibung von Teilchen und deren Eigenschaften. Es vereint die drei fundamentalen Kräfte der elektromagnetischen, der schwachen und der starken Kernkraft zu einer einheitlichen Theorie und wurde in zahlreichen Experimenten bestätigt, zuletzt durch die Entdeckung des Higgs-Bosons im Jahre 2012. Die Parameter des Standardmodells werden mit immer höherer Genauigkeit bestimmt.\\

Trotz seines Erfolgs und seiner Zuverlässigkeit besteht allgemein der Konsens, dass das Standardmodell keine vollständige Theorie ist, da beispielsweise die vierte fundamental Kraft der Gravitation, die im Vergleich zu den anderen Kräften extrem schwach ist (auch bekannt als das Hierarchie-Problem), nicht im Standardmodell enthalten ist. Außerdem gibt es in der Kosmologie experimentelle Hinweise auf eine neue Form von Materie, sogenannte Dunkle Materie. Sie wechselwirkt lediglich über die Gravitation und möglicherweise über die schwache Kernkraft. Physikalische Phänomene, die nicht vom Standardmodell beschrieben werden, werden unter dem Namen Physik jenseits des Standardmodells zusammengefasst. Ein Beweis für Physik jenseits des Standardmodells steht noch immer aus und ist eine der wichtigsten Motivationen für Experimente in der Teilchenphysik.\\

Der größte und fortschrittlichste Teilchenbeschleuniger der Welt ist der Large Hadron Collider (LHC) am Forschungszentrum CERN in Genf. Er ging im Jahre 2010 in Betrieb und erzeugte Teilchenkollisionen mit einer bis dato unerreichten Schwerpunktsenergie von 7\,TeV. Dabei wurden ausreichend Daten gesammelt um die Existenz des Higgs-Bosons zu beweisen. Zu den neuen Aufgaben des LHCs gehören nun beispielsweise die Suche nach weiteren Higgs-Bosonen oder allgemein die Suche nach Physik jenseits des Standardmodells. Um das zu erreichen wurde die Schwerpunktsenergie auf 13\,TeV erhöht, wodurch auch neue, bisher unzugängliche seltene Prozesse untersucht werden können.\\

Ein Bereich von besonderem Interesse für Suchen nach Physik jenseits des Standardmodells is die Produktion von Paaren von Eichbosonen der elektroschwachen Wechselwirkung mit hohem transversalem Impuls. Hier würde sich die Existenz von neuen, exotischen Teilchen besonders stark bemerkbar machen. Der dominante Produktionskanal für diesen Prozess findet über die Kopplung von drei Eichbosonen, die sogenannte dreifache Eichkopplung, statt. Die Effekte von möglicher Physik jenseits des Standardmodells können durch Änderungen dieser Kopplung beschrieben werden. Die veränderten Kopplungen werden anomale dreifache Eichkopplungen genannt. Eine Möglichkeit, diese anomalen Kopplungen modelunabhängig zu beschreiben, bietet der Ansatz über effektive Feldtheorien. Dabei wird eine Energieskala $\Lambda$ eingeführt, die Größenordnung, bei der die neue Physik auftritt, beschreibt. Dabei gilt die Annahme, dass die Energieskala $\Lambda$ viel größer als die in den Teilchenkollisionen erreichte Energie ist. Die Stärke der anomalen Kopplungen wird durch die drei Parameter \Tcwww \ , \Tccw \ und \Tcb \ beschrieben.