\chapter*{Zusammenfassung}
Seit seiner Entwicklung zwischen 1960 und 1970 ist das Standardmodell der Teilchenphysik bis heute die erfolgreichste und zuverlässigste Theorie zur Beschreibung von Teilchen und deren Eigenschaften. Es vereint die drei fundamentalen Kräfte der elektromagnetischen, der schwachen und der starken Kernkraft zu einer einheitlichen Theorie und wurde in zahlreichen Experimenten bestätigt, zuletzt durch die Entdeckung des Higgs-Bosons im Jahre 2012.\\

Trotz seines Erfolgs und seiner Zuverl\"assigkeit besteht allgemein der Konsens, dass das Standardmodell keine vollständige Theorie ist. So ist beispielsweise die vierte fundamental Kraft der Gravitation, die im Vergleich zu den anderen Kräften extrem schwach ist (auch bekannt als das Hierarchie-Problem), nicht im Standardmodell enthalten ist. Außerdem gibt es in der Kosmologie experimentelle Hinweise auf eine neue Form von Materie, sogenannte Dunkle Materie. Sie wechselwirkt lediglich über die Gravitation und möglicherweise über die schwache Kernkraft. Die Suche nach Physik jenseits des Standardmodells ist eine der wichtigsten Motivationen für Experimente in der Teilchenphysik.\\

Der größte und fortschrittlichste Teilchenbeschleuniger der Welt ist der Large Hadron Collider (LHC) am Forschungszentrum CERN in Genf. Er ging im Jahre 2010 in Betrieb und erzeugte Teilchenkollisionen mit einer bis dato unerreichten Schwerpunktsenergie von 7-8\,TeV. Dabei wurden ausreichend Daten gesammelt um die Existenz des Higgs-Bosons zu beweisen. Zu den neuen Aufgaben des LHCs gehören nun beispielsweise die Suche nach weiteren Higgs-Bosonen oder allgemein die Suche nach Physik jenseits des Standardmodells. Die Erhöhung der Schwerpunktsenergie auf 13\,TeV im Jahre 2013 ermöglicht neue Analysen, die bisher unzugängliche seltene Prozesse untersuchen.\\

Ein Bereich von besonderem Interesse für Suchen nach Physik jenseits des Standardmodells ist die Produktion von Paaren von Eichbosonen der elektroschwachen Wechselwirkung mit hohem transversalem Impuls. Hier würde sich die Existenz von neuen, exotischen Teilchen besonders stark bemerkbar machen. Ein möglicher Produktionskanal für diesen Prozess findet über die Kopplung von drei Eichbosonen, die sogenannte dreifache Eichkopplung, statt. Die Effekte von Physik jenseits des Standardmodells können durch Änderungen dieser Kopplung modellunabhängig beschrieben werden. Die veränderten Kopplungen werden anomale dreifache Eichkopplungen (aTGC) genannt. Eine Möglichkeit für eine solche modelunabhängige Beschreibung bietet der Ansatz über effektive Feldtheorien. Dabei wird eine Energieskala $\Lambda$ eingeführt, die die Größenordnung, bei der die neue Physik auftritt, beschreibt. Der Ansatz der effektiven Feldtheorie ist nur gültig, wenn die Energieskala $\Lambda$ viel größer als die in den Teilchenkollisionen erreichte Energie ist. Dieser Ansatz bietet somit die Möglichkeit, nach neuer Physik bei bisher unerreichten Energien zu suchen. Die Stärke der anomalen Kopplungen wird durch folgende drei aTGC-Parameter charakterisiert, welche im Standardmodell gleich Null sind:
\begin{equation*}
\cwww \ , \ccw \ , \cb \ .
\end{equation*}
Das Ziel dieser Arbeit besteht darin, diese Parameter zu messen bzw. Ausschlussgrenzen auf diese Parameter zu bestimmen. Dafür wird ein Datensatz verwendet, der im Jahre 2015 mit dem CMS Detektor bei einer Schwerpunktsenergie von $\sqrt{s}=13$\,TeV aufgenommen wurde.\\

Basierend auf den Zerfallsarten der beiden Bosonen werden bei der Diboson-Produktion verschiedene Kanäle unterschieden. Jedes der Bosonen kann entweder in zwei Leptonen oder in zwei Quarks zerfallen. In dieser Analyse wird der semileptonische Kanal untersucht, wobei ein W-Boson leptonisch und ein weiteres W- oder Z-Boson hadronisch zerfällt. Dieser Kanal bietet ein höheres Verzweigungsverhältnis als der voll-leptonische Kanal und weist geringere Untergrundbeiträge als der voll-hadronische Kanal auf. Die Signatur dieser Zerfälle weist ein geladenes Lepton und ein Neutrino aus dem leptonischen Zerfall eines W-Bosons sowie zwei Teilchenschauer aus dem hadronischen Zerfall des anderen W- oder Z-Bosons auf. Um die Sensitivität auf anomale Kopplungen weiter zu erhöhen werden nur Ereignisse mit hohem Transversalimpuls betrachtet. Dies führt dazu, dass die beiden Teilchenschauer aus dem hadronischen zerfallenden Boson nicht mehr separat sondern als einzelner Schauer mit größerem Radius rekonstruiert werden.\\

Die Masse des rekonstruierten Teilchenschauers wird verwendet, um eine Signal- und eine Seitenbandregion zu  definieren. Aus den Ereignissen in der Seitenbandregion wird der Untergrund in der Signalregion abgeschätzt. Aufgrund der unterschiedlichen Beiträge der aTGC-Parameter zu WW- und WZ-Ereignissen wird die Signalregion in eine von WW-Ereignissen und eine von WZ-Ereignissen dominierte Region aufgeteilt. \\

Die Effekte der anomalen Kopplungen machen sich besonders bei hohen invarianten Massen des WW- bzw. WZ-Paares bemerkbar, weshalb das Spektrum dieser Variable für die Suche verwendet wird. Sie sind als Funktionen implementiert, welche die aTGC-Paramter explizit enthalten. Dies ermöglicht es, verschiedene Kombinationen von aTGC-Parametern zu untersuchen sowie Interferenzeffekte zwischen dem Standardmodell-Prozess und dem aTGC-Prozess sowie Interferenzeffekte zwischen den verschiedenen aTGC-Beiträgen zu berücksichtigen.\\

\begin{table}
	\centering
	\caption[Erwartete und beobachtete Ausschlussgrenzen bei einem Konfidenzniveau von 95\%]{Erwartete und beobachtete eindimensionale Ausschlussgrenzen auf die aTGC-Parameter bei einem Konfidenzniveau von 95\%.}
	\label{tab:intro_ger:1d}
	\begin{tabular}{ccc}
	\hline
	aTGC              &     erwartete Grenze & beobachtete Grenze\\
	\hline
	$\frac{c_{WWW}}{\Lambda ^2}$~(TeV$^{-2}$) &  [-8.73 , 8.70] &  [-9.46 , 9.42] \\
	$\frac{c_{W}}{\Lambda ^2}$~(TeV$^{-2}$)   &  [-11.7 , 11.1] &  [-12.6 , 12.0] \\
	$\frac{c_{B}}{\Lambda ^2}$~(TeV$^{-2}$)   & [-54.9 , 53.3] &  [-56.1 , 55.4] \\
	\hline
	\end{tabular}
\end{table}

Um Ausschlussgrenzen auf die aTGC-Parameter zu bestimmen werden die beobachteten Daten mit der Untergrundabschätzung und dem Signal-Modell verglichen. Dazu wird eine aus der Untergrundabsch\"atzung, dem Signalmodell und den systematischen Unsicherheiten eine Likelihood-Funktion aufgestellt. Zun\"achst wird derjenige aTGC-Parameterwert bestimmt, der diese Likelihood-Funktion minimiert, der sogenannte Best-Fit-Wert. Anschließend wird die Likelihood-Funktion für verschiedene aTGC-Parameterwert bezüglich den systematischen Unsicherheiten minimiert und die Differenz mit dem Wert der Likelihood-Funktion des Best-Fit-Wertes gebildet. Aus der resultierenden Verteilung werden Ausschlussgrenzen bei einem Konfidenzniveau von 95\% ermittelt. Unterschiedliche Szenarien werden betrachtet, bei denen entweder ein oder zwei aTGC-Parameter von Null verschieden sind, was zu ein- oder zweidimensionalen Ausschlussgrenzen führt. Die ermittelten eindimensionalen Ausschlussgrenzen sind in Tabelle~\ref{tab:intro_ger:1d} zusammengefasst und sind mit dem Standardmodell kompatibel. Ein Beispiel für die zweidimensionalen Ausschlussgrenzen ist in Abbildung~\ref{fig:intro_ger:2d} dargestellt. Wie hier zu sehen ist, befinden sich die beobachteten Grenzen nahe bei den erwarteten. Außerdem befindet sich der vom Standardmodell erwartete Wert innerhalb der Ausschlussgrenzen, d.h. es wurden keine Anzeichen für anomale Kopplungen gemessen.\\

Die in dieser Thesis präsentierten Ausschlussgrenzen auf die aTGC-Parameter sind die ersten Ergebnisse zu anomalen dreifachen Eichkopplungen bei $\sqrt{s}=13$\,TeV, die von der CMS Kollaboration veröffentlicht wurden. Die ermittelten Grenzen sind vergleichbar mit denen früherer Analysen, trotz des vergleichbar kleinen verwendeten Datensatzes. Die Verwendung des gesamten im Jahre 2016 aufgenommenen Datensatzes verspricht eine deutliche Verbesserung der bisherigen Grenzen.


\begin{figure}
    \centering
    \resizebox{0.8\columnwidth}{!}
    {%
    \includegraphics[height=0.3\textheight]{../plots/bkg/mlvj/WW_mu_cwww.pdf}
    \caption[Vergleich der gemessenen Daten mit der Untergrundabschätzung sowie zweidimensionale Ausschlussgrenzen auf \Tccw -\Tcb]{Vergleich der gemessenen Daten mit der Untergrundabschätzung im Myon-Kanal, WW-Region, im invarianten Massespektrum des Dibosonsystems. Zusätzlich ist die erwartete Signalfunktion für einen von Null verschiedenen aTGC-Parameter $\Tcwww=12$\,TeV$^{-2}$ (violett gestrichelt) eingezeichnet. Das untere Viertel des Schaubilds zeigt die Differenz zwischen Untergrundabschätzung und beobachteten Daten, normiert auf die Unsicherheit der Daten.}
    \label{fig:intro_ger:WW}
    }
\end{figure}

\begin{figure}
	\centering
	\subfigure[]{
		\includegraphics[width=0.5\textwidth]{../plots/results/cwww_cb_2dlimit_deltaNLL.pdf}
	}%
	\subfigure[]{
		\includegraphics[width=0.5\textwidth]{../plots/results/ccw_cb_2dlimit_deltaNLL.pdf}
	}		
	\caption[Zweidimensionale Likelihoodverteilungen für zwei mögliche Kombinationen von aTGC-Parametern]{Zweidimensionale Likelihoodverteilungen für zwei der drei möglichen Kombinationen von aTGC-Parametern, \Tcwww -\Tcb \ (a) und \Tccw -\Tcb \ (b). Die beobachteten Ausschlussgrenzen bei einem Konfidenzniveau von 95\% sind als schwarze Kontur dargestellt. Diese liegen nahe bei den erwarteten Grenzen (grün gestrichelt). Außerdem gezeigt sind die erwarteten Grenzen bei einem Konfidenzniveau von 68\% (blau gestrichelt) und 99\% (rot gestrichelt).}
	\label{fig:intro_ger:2d}	
\end{figure}
